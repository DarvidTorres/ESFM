\documentclass[11pt]{article}

%\usepackage[utf8]{inputenc}
%\usepackage[T1]{fontenc}
\usepackage{xcolor,graphicx}
\graphicspath{{images/}}
\usepackage[top=0.6in,bottom=0.6in,right=1in,left=1in]{geometry}
\usepackage{amsfonts, amssymb, amsmath, amsthm, enumitem}
\usepackage{mathtools}
\usepackage{braket}

%Dark mode
%\pagecolor[rgb]{0,0,0} %black
%\color[rgb]{1,1,1} %white

%the code below manipulates space around \align environment
%\usepackage{etoolbox}
%\newcommand{\zerodisplayskips}{%
%\setlength{\abovedisplayskip}{10pt}%
%\setlength{\belowdisplayskip}{-10pt}%
%\setlength{\abovedisplayshortskip}{10pt}%
%\setlength{\belowdisplayshortskip}{-10pt}}
%\appto{\normalsize}{\zerodisplayskips}
%\appto{\small}{\zerodisplayskips}
%\appto{\footnotesize}{\zerodisplayskips}

%Conjuntos de números
\newcommand{\N}{\mathbb{N}}
\newcommand{\Z}{\mathbb{Z}}
\newcommand{\Q}{\mathbb{Q}}
\newcommand{\R}{\mathbb{R}}

%Shorter comands
\newcommand{\defined}{\coloneqq}
\let\epsilon\varepsilon
\let\for\forall
\let\oldemptyset\emptyset
\let\emptyset\varnothing
\let\set\Set

%bold all lists$
\setlist[enumerate]{font=\bfseries}

\setlength{\parindent}{0pt} %no indent for the document
\setlength{\parskip}{1em} %add space between paragraphs
\pagestyle{empty}

\begin{document}

\title{\vspace{-2cm}Cálculo I}
\author{Darvid \\ \texttt{darvid.torres@gmail.com}}
\date{\today}
\maketitle
\thispagestyle{empty}

\section*{Números reales}

Existe un conjunto llamado conjunto de los números reales, denotado por $\mathbb{R}$. A los elementos de este conjunto los llamaremos números reales. Este conjunto está dotado con dos operaciones binarias:

\begin{center} \begin{tabular}{rl}
    Suma $+:$ & $ \mathbb{R} \times \mathbb{R} \to \mathbb{R} $ \\
    & $ (m, n) \mapsto m+n $\\
    \\
    Multiplicación $\cdot:$ & $ \mathbb{R} \times \mathbb{R} \to \mathbb{R} $ \\
    & $ (m, n) \mapsto m\cdot n $
\end{tabular} \end{center}

Las cuales satisfacen los siguientes:

\subsection*{Axiomas de campo}

\begin{enumerate}[label=A\arabic*., font=\bfseries]
    \item La suma es conmutativa. Esto significa que para cualesquiera números reales $m$ y $n$ se verifica que: $ m+n = n+m $.
    \item La suma es asociativa. Esto significa que para cualesquiera números reales $m$, $n$ y $l$ se verifica que: $ m+(n+ l) = (m+ n)+l $.
    \item Elemento neutro para la suma. Existe un número real llamado elemento neutro para la suma o cero, denotado por $0$, el cual satisface la siguiente condición: $ m+0=m,\forall m \in \mathbb{R} $.
    \item Inverso aditivo. Para cada número real $m$ existe un número real llamado inverso aditivo de $m$, denotado por $-m$ (menos $m$); la propiedad que caracteriza a este elemento es: $ m + (-m) = 0 $.
    \item La multiplicación es conmutativa. Esto significa que para cualesquiera números reales $m$ y $n$ se verifica que: $ m \cdot n = n \cdot m $.
    \item La multiplicación es asociativa. Esto significa que para cualesquiera números reales $m$, $n$ y $l$ se verifica que: $ m \cdot (n \cdot l) = (m \cdot n) \cdot l $.
    \item Elemento identidad para la multiplicación. Existe un número real distinto de cero, llamado elemento identidad para la multiplicación o uno, denotado por $1$, que satisface la siguiente condición: $ m \cdot 1 = m,\forall m \in \mathbb{R} $.
    \item Inverso multiplicativo. Para cada número real $m$ distinto de cero existe un número real llamado inverso multiplicativo de $m$, denotado por $m^{-1}$, este elemento tiene la siguiente propiedad:$m \cdot m^{-1} = 1$.
    \item Distribución de la multiplicación sobre la suma. Para cualesquiera números reales $m$, $n$ y $l$ se verifica que: $ m \cdot (n+l)=m \cdot n+m \cdot l $.
\end{enumerate}

\subsection*{Lista de ejercicios 1 (LE1)}
    \begin{enumerate}[label=\alph*),font=\bfseries]
        \item Demuestre que el elemento neutro para la suma es único.
        \item Demuestre que el elemento identidad para la multiplicación es único.
        \item Sea $m$ un número real arbitrario pero fijo, demuestre que el inverso aditivo de $m$ es único.
        \item Sea $m$ un número real distinto de cero, demuestre que el inverso multiplicativo de $m$ es único.
        \item Demuestre que $-0 = 0$.
        \item Sea $m$ un número real arbitrario pero fijo, demuestre que $m \cdot 0 = 0$.
        \item Si $m$ y $n$ son números reales tales que $ m \cdot n = 0 $, demuestre que $m=0$ o $n=0$.
        \item Sea $m$ un número real arbitrario pero fijo, demuestre que: $(-1) \cdot m =-m $.
        \item Sean $m$ y $n$ números reales, demuestre que: $ (-m) \cdot n = -(m \cdot n) $.
        \item Sea $m$ un número real arbitrario pero fijo, demuestre que: $-(-m)=m$.
        \item Sean $m$ y $n$ números reales, demuestre que: $ (-m) \cdot (-n) = m \cdot n $.
        \item Sea $m$ un número real arbitrario pero fijo, demuestre que: $(-1) \cdot (-m)=m$.
        \item Sea $m$ un número real distinto de cero; demuestre que: $\left( m^{-1} \right )^{-1}=m$.
        \item Sean $m$ y $n$ números reales distintos de cero, demuestre que: $(m \cdot n)^{-1}=m^{-1} \cdot n^{-1}$.
    \end{enumerate}

\subsubsection*{Demostración:}
\begin{enumerate}[label=\alph*), font=\bfseries]
    \item Supongamos que existen 0 y $\tilde{0}$ números reales tales que $m+0 = m$ y $m+\tilde{0} = m$. Notemos que: \begin{align*}
        0 &=m+(-m) && \text{Por A4}\\
        &=\left( m+\tilde{0} \right)+\left(-m\right) && \text{Por hipótesis}\\
        &=\left( \tilde{0}+m \right)+\left(-m\right) && \text{Por A1}\\
        &=\tilde{0} + \bigl( m + \left(-m \right)\bigr) && \text{Por A2}\\
        &=\tilde{0} + 0 && \text{Por A4}\\
        &=\tilde{0} && \text{Por A3}
        \end{align*} \qed

    \item Supongamos que existen $1$ y $\tilde{1}$ números reales tales que$m\cdot 1=m$ y $m\cdot\tilde{1}=m$. Notemos que:
        \begin{align*}
        1 &= m \cdot m^{-1} && \text{Por A8}\\
        &= \left( m \cdot \tilde{1} \right) \cdot m^{-1} && \text{Por hipótesis}\\
        &= \left( \tilde{1} \cdot m \right) \cdot m^{-1} && \text{Por A5}\\
        &= \tilde{1} \cdot \left( m \cdot m^{-1} \right) && \text{Por A6}\\
        &= \tilde{1} \cdot 1 && \text{Por A8}\\
        &= \tilde{1} && \text{Por A7}
        \end{align*} \qed

    \item Supongamos que existen $-m$ y $-\tilde{m}$ números reales tales que $m + \left(-m\right) = 0$ y $m + \left(- \tilde{m}\right) = 0$. Notemos que:
        \begin{align*}
        -m &= -m+0 && \text{Por A3}\\
        &= 0+\left(-m\right) && \text{Por A1}\\
        &= \bigl(m+\left(-\tilde{m} \right)\bigr)+\left(-m\right) && \text{Por (2)}\\
        &= \bigl(\left(-\tilde{m} \right)+m\bigr)+\left(-m\right) && \text{Por A1}\\
        &= \left(-\tilde{m} \right)+\bigl(m+\left(-m\right)\bigr) && \text{Por A2}\\
        &= \left(-\tilde{m} \right)+0 && \text{Por A8}\\
        &= -\tilde{m} && \text{Por A3}
        \end{align*} \qed

    \item Supongamos que existen $m^{-1}$ y $\tilde{m}^{-1}$ números reales, distintos de cero, tales que $m \cdot m^{-1} = 1$ y $m \cdot \tilde{m}^{-1} = 1$. Notemos que:
        \begin{align*}
        m^{-1} &= m^{-1} \cdot 1 && \text{Por A7} \\
        &= m^{-1} \cdot \left(m \cdot \tilde{m}^{-1} \right) && \text{Por hipótesis} \\
        &= \left( m^{-1} \cdot m \right) \cdot \tilde{m} ^{-1} && \text{Por A6} \\
        &= \left(m \cdot m^{-1} \right) \cdot \tilde{m}^{-1} && \text{Por A5} \\
        &= 1 \cdot \tilde{m}^{-1} && \text{Por hipótesis} \\
        &= \tilde{m}^{-1} \cdot 1 && \text{Por A5} \\
        &= \tilde{m}^{-1} && \text{Por A7}
        \end{align*} \qed
    
    \item Por A3 se verifica que $0 + 0 = 0$, y por A4 que $0 + (-0) = 0$. Además, por (c) de LE1, tenemos que el inverso aditivo de cada número real es único, entonces debe ser el caso que $-0 = 0$. \mbox{}\hfill  $\square$
    
    \item Notemos que:
    \begin{align*}
        m\cdot0&=m\cdot0+0 && \text{Por A3}\\
        &=m\cdot0+\bigl(m+\left(-m\right)\bigr) && \text{Por A4}\\
        &=m\cdot0+\bigl(m\cdot1+\left(-m\right)\bigr) && \text{Por A7}\\
        &=\left(m\cdot0+m\cdot1\right)+\left(-m\right) && \text{Por A2}\\
        &=\bigl(m\cdot\left(0+1\right)\bigr)+\left(-m\right) && \text{Por A9}\\
        &=m\cdot1+\left(-m\right) && \text{Por A3}\\
        &=m+\left(-m\right) && \text{Por A7}\\
        &=0 && \text{Por A4}\\
    \end{align*} \qed
    
    \item Supongamos que $m$ es distinto de $0$. Notemos que:
    \begin{align*}
        n &= n \cdot 1	&& \text{Por A7} \\
        &= n \cdot  \left(m \cdot m^{-1}  \right) 	&& \text{Por A8} \\
        &= \left(n \cdot m \right)  \cdot m^{-1}	&& \text{Por A6} \\
        &= \left(m \cdot n \right)  \cdot m^{-1}	&& \text{Por A5} \\
        &= 0 \cdot m^{-1}	&& \text{Por hipótesis}\\
        &= m^{-1} \cdot 0	&& \text{Por A5}\\
        &= 0 && \text{Por (f) de LE1}
    \end{align*} \qed
    
    \item Notemos que:
    \begin{align*}
        -m&=-m+0 && \text{Por A3} \\
        &=-m+m \cdot 0 && \text{Por (f) de LE1} \\
        &=-m+m \cdot  \bigl( 1+ \left( -1 \right)  \bigr) && \text{Por A4} \\
        &=-m+ \bigl( m \cdot 1+m \cdot  \left( -1 \right)  \bigr) && \text{Por A9} \\
        &=-m+ \bigl( m+m \cdot  \left( -1 \right)  \bigr) && \text{Por A8} \\
        &= \left( -m+m \right) +m \cdot  \left( -1 \right) && \text{Por A2} \\
        &=m+ \left( -m \right) +m \cdot  \left( -1 \right) && \text{Por A1} \\
        &=0+m \cdot  \left( -1 \right) && \text{Por A4} \\
        &=m \cdot  \left( -1 \right) + 0 && \text{Por A1} \\
        &=m \cdot  \left( -1 \right) && \text{Por A3} \\
        &= \left( -1 \right)  \cdot m && \text{Por A5}
    \end{align*} \qed
    
    \item Notemos que:
    \begin{align*}
        (-m) \cdot n &= \bigl( \left(-1 \right) \cdot m \bigr) \cdot n && \text{Por (h) de LE1}\\
        &= (-1) \cdot (m \cdot n) && \text{Por A6}\\
        &= -(m \cdot n) && \text{Por (h) de LE1}\\
    \end{align*} \qed

    \item Sea $m$ un número real arbitrario pero fijo. Por A4 se verifica que $m + (-m) = 0$, y por A1 tenemos que $(-m) + m = 0$, de esta igualdad se sigue que $m$ es inverso aditivo de $(-m)$. Por A4 se verifica que $(-m) + \bigl(-(-m)\bigr) = 0$, y por (c) de LE1, sabemos que el inverso aditivo de cada número real es único. Entonces, debe ser el caso que $-(-m) = m$.
    \qed

    \item Notemos que:
    \begin{align*}
        (-m) \cdot (-n) &= (-m) \cdot \bigl( (-1) \cdot n \bigr) && \text{Por (h) de LE1}\\
        &= \bigl( (-m) \cdot (-1) \bigr) \cdot n && \text{Por A6}\\
        &= \bigl( (-1) \cdot (-m) \bigr) \cdot n && \text{Por A5}\\
        &= -(-m) \cdot n && \text{Por (h) de LE1}\\
        &= m \cdot n && \text{Por (j) de LE1}\\
    \end{align*}
    \qed

    \item Notemos que:
    \begin{align*}
        (-1) \cdot (-m) &= 1 \cdot m && \text{Por (k) de LE1}\\
        &= m \cdot 1 && \text{Por A5}\\
        &= m && \text{Por A7}\\
    \end{align*}

    \item Sea $m$ un número real distinto de cero. Por A8 sabemos que $m \cdot m^{-1}=1$, y por A5 tenemos que $m^{-1} \cdot m=1$, de esta igualdad se sigue que $m$ es inverso multiplicativo de $m^{-1}$. Por A8 se verifica que $ m^{-1} \cdot \left( m^{-1} \right)^{-1} =1$, y por (d) de LE1, sabemos que el inverso multiplicativo de cada número real es único. Entonces, debe ser el caso que $\left( m^{-1} \right )^{-1}=m$.
    \qed

    \item Sean $m$ y $n$ números reales distintos de cero. Notemos que:
    \begin{align*}
        \left(m \cdot n \right) \cdot  \left(m^{-1} \cdot n^{-1}  \right)	&=	 \left( \left(m \cdot n \right) \cdot m^{-1}  \right) \cdot n^{-1}  	&& \text{Por A6}\\
    &=	 \left( \left(n \cdot m \right) \cdot m^{-1}  \right) \cdot n^{-1}  	&& \text{Por A5}\\
    &=	 \Bigl(n \cdot  \left(m \cdot m^{-1}\right) \Bigr) \cdot n^{-1}	&& \text{Por A6}\\
    &=	 \left(n \cdot 1 \right) \cdot n^{-1}	&& \text{Por A8}\\
    &=	n \cdot n^{-1}	&& \text{Por A7}\\
    &=	1	&& \text{Por A8}
    \end{align*} \\
    De la igualdad anterior, sigue que $\left(m^{-1} \cdot n^{-1} \right)$ es inverso multiplicativo de $\left( m \cdot n \right)$. Además, por (d) de LE1, sabemos que el inverso multiplicativo de cada número real es único. Entonces, debe ser el caso que $\left(m^{-1} \cdot n^{-1} \right) = \left( m \cdot n \right)^{-1}$ \qed
\end{enumerate}

\textbf{Notación:}

\begin{itemize}
    \item Si $m$ y $n$ son números reales, representaremos con el símbolo $m-n$ a la suma: $m+ (-n)$.
    \item Si $m$ y $n$ son números reales y $n$ es distinto de cero, representaremos con el símbolo $ \frac{m}{n}$ al número $m \cdot n^{-1} $.
    \item Si $m_1$, $m_2$ y $m_3$ son números reales, representaremos con el símbolo $m_1+m_2+ m_3$ a cualquiera de las sumas $m_1+ \left(m_2+ m_3 \right)$ o $\left(m_1+ m_2 \right)+ m_3$.
\end{itemize}

%LISTA DE EJERCICIOS 2

\subsection*{Lista de ejercicios 2 (LE2)}

Sean $a$, $b$, $c$ y $d$ números reales, demuestre lo siguiente:

\begin{enumerate}[label=\alph*),font=\bfseries]
    \item $a \cdot \frac{c}{b} = \frac{ac}{b}$, si $b \neq 0$
    \item $\frac{a}{b} = \frac{ac}{bc}$, si $b,c \neq 0$
    \item $\frac{a}{b} \pm \frac{c}{d} = \frac{ad \pm bc}{bd} $, si $b, d \neq 0$
    \item $\frac{a}{b} \cdot \frac{c}{d} = \frac{ac}{bd}$, si $b, d \neq 0$
    \item $\frac{\frac{a}{b}}{\frac{c}{d}} = \frac{ad}{bc}$, si $b, c, d \neq 0$
\end{enumerate}

\subsubsection*{Demostración}

\begin{enumerate}[label=\alph*),font=\bfseries]

%A

\item Notemos que:
\begin{align*}
    a \cdot \frac{c}{b} &= a \cdot \left( c \cdot b^{-1} \right) && \text{Por notación}\\
    &= \left( a \cdot c \right) \cdot b^{-1} && \text{Por A6}\\
    &= \frac{ac}{b} && \text{Por notación}\\
\end{align*}
\qed

%B

\item Notemos que:
\begin{align*}
    \frac{ac}{bc} &= a \cdot c \cdot \left( bc \right)^{-1} && \text{Por notación}\\
    &= a \cdot c \cdot b^{-1} \cdot c^{-1} && \text{Por (m) de LE1}\\
    &= a \cdot b^{-1} \cdot c \cdot c^{-1} && \text{Por A5}\\
    &= a \cdot b^{-1} \cdot 1 && \text{Por A8}\\
    &= a \cdot b^{-1} && \text{Por A7}\\
    &= \frac{a}{b} && \text{Por notación}
\end{align*}
\qed

%C

\item Notemos que:

\begin{align*}
\frac{a}{b} \pm \frac{c}{d}  &=	a \cdot b^{-1} \pm c \cdot d^{-1} && \text{Por notación}\\
&=	\left( a \cdot 1 \right)   \cdot b^{-1} \pm \left( c \cdot 1 \right) \cdot d^{-1} && \text{Por A7}\\
&=	\Bigl( a \cdot  \left( d \cdot d^{-1} \right) \Bigr) \cdot b^{-1} \pm \Bigl( c \cdot  \left( b \cdot b^{-1}  \right)  \Bigr)  \cdot d^{-1} && \text{Por A8}\\
&=	\Bigl(  \left( a \cdot d \right) \cdot d^{-1} \Bigr) \cdot b^{-1} \pm \Bigl(  \left( c \cdot b \right) \cdot b^{-1} \Bigr) \cdot d^{-1} && \text{Por A6}\\
&=	\left( a \cdot d \right)  \cdot  \left( d^{-1} \cdot b^{-1}  \right) \pm \left( c \cdot b \right)  \cdot  \left( b^{-1} \cdot d^{-1}  \right) && \text{Por A6}\\
&=	\left( a \cdot d \right)  \cdot  \left( b^{-1} \cdot d^{-1}  \right) \pm \left( c \cdot b \right)  \cdot  \left( b^{-1} \cdot d^{-1}  \right) && \text{Por A5}\\
&=	\left( b^{-1} \cdot d^{-1}  \right) \cdot \left( a \cdot d \pm c \cdot b \right) && \text{Por A9}\\
&=	\left( a \cdot d \pm c \cdot b \right) \cdot  \left( b^{-1} \cdot d^{-1} \right) && \text{Por A5}\\
&=	\left( a \cdot d \pm c \cdot b \right) \cdot  \left( b \cdot d \right)^{-1} && \text{Por (m) de LE1}\\
&=	\left( a \cdot d \pm b \cdot c \right) \cdot \left( b \cdot d \right)^{-1} && \text{Por A5}\\
&=	\frac{ad \pm bc}{bd} && \text{Por notación}\\
\end{align*}
\qed

%D

\item Notemos que

\begin{align*}
    \frac{a}{b} \cdot \frac{c}{d} &= \left( a \cdot b^{-1} \right) \cdot \left( c \cdot d^{-1} \right) && \text{Por notación}\\
        &= a \cdot \Bigl( b^{-1} \cdot \left( c \cdot d^{-1} \right) \Bigr) && \text{Por A6}\\
            &= a \cdot \Bigl( c \cdot \left( b^{-1} \cdot d^{-1} \right) \Bigr) && \text{Por A5}\\
	&= \left( a \cdot c \right) \cdot \left( b^{-1} \cdot d^{-1} \right) && \text{Por A6}\\
	&= \left( a \cdot c \right) \cdot \left( b \cdot d \right)^{-1} && \text{Por (m) de LE1}\\
	&= \frac{ac}{bd} && \text{Por notación}\\
\end{align*}
\qed

\pagebreak
%E

\item Notemos que:

\begin{align*}
    \frac{\frac{a}{b}}{\frac{c}{d}} &= \frac{\left( a \cdot b^{-1} \right)}{\left( c \cdot d^{-1} \right)} && \text{Por notación}\\
    &= \left( a \cdot b^{-1} \right) \cdot \left( c \cdot d^{-1} \right)^{-1} && \text{Por notación}\\
    &= \left( a \cdot b^{-1} \right) \cdot \left( c^{-1} \cdot \left( d^{-1} \right) ^{-1} \right) && \text{Por (m) de LE1}\\
    &= \left( a \cdot b^{-1} \right) \cdot \left( c^{-1} \cdot d \right) && \text{Por (l) de LE1}\\
    &= \left( a \cdot b^{-1} \right) \cdot \left( d \cdot c^{-1} \right) && \text{Por A5}\\
    &= \frac{a}{b} \cdot \frac{d}{c} && \text{Por notación}\\
    &= \frac{ad}{bc} && \text{Por (d) de LE2}\\
\end{align*}
\qed

\end{enumerate}

\subsection*{Axiomas de orden del conjunto de los números reales}

Existe un subconjunto del conjunto de los números reales llamado conjunto de los números reales positivos, denotado con el símbolo $\mathbb{R}^+$. A los elementos de este conjunto los llamaremos números reales positivos.\\
El conjunto $\mathbb{R}^+$ satisface los siguientes axiomas:

\begin{enumerate}[label=O\arabic*), font=\bfseries]
    \item Si $m, n \in \mathbb{R}^+$, entonces $m + n \in \mathbb{R}^+$.
    \item Si $m, n \in \mathbb{R}^+$, entonces $m \cdot n \in \mathbb{R}^+$.
    \item Para cada número real $m$ se cumple una y sólo una de las siguientes condiciones: \label{tricotomía}
	\begin{enumerate}[label=\roman*), font=\bfseries]
    \item $m \in \mathbb{R}^+$.
    \item $m = 0$.
    \item $-m \in \mathbb{R}^+$.\
    \end{enumerate}
\end{enumerate}

\textbf{Definición:} Sean $a$ y $b$ números reales, decimos que:
    \begin{enumerate}
        \item $a$ es menor que $b$ o que $b$ es mayor que $a$ y escribimos $a<b$ o $b>a$, si $b-a \in \mathbb{R}^+$.
    \item $a$ es menor que o igual que $b$ o que $b$ es mayor o igual que $a$, y escribimos $a \leq b$ o $b \geq a$, si $b - a \in \mathbb{R}^+$ o $a = b$.
    \end{enumerate}

\textbf{Notación:} Sean $a$, $b$ y $c$ números reales, utilizaremos la notación $a<b<c$ para indicar que $a<b$ y $b<c$.

\pagebreak

\subsection*{Lista de Ejercicios 3 (LE3)}

%LISTA DE EJERCICIOS 3

Sean $a$, $b$, $c$ y $d$ números reales, demuestre lo siguiente:

\begin{enumerate}[label=\alph*),font=\bfseries]
    \item $1 \in \R^+$.
    \item $a \in \mathbb{R}^+$ si y solo si $a>0$. %A
    \item $-1<0$. %B
    \item Si $a<b$ y $c \leq d$, entonces $a+c<b+d$. %C
    \item Si $a<b$ y $0<c$, entonces $ac<bc$. %D
    \item Si $a<b$ y $c<0$, entonces $ac>bc$. %E
    \item $a \in \mathbb{R}^+$ si y solo si $-a<0$. %F
    \item $a<b$ si y solo si $-a>-b$. %G
    \item Si $a>0$, entonces $\frac{1}{a}>0$. %H
    \item Si $a<b$ y $b<c$, entonces $a<c$. %I
    \item Si $0 \leq a<b$ y $0 \leq c<d$, entonces $ac<bd$. %J
    \item Si $a<b$ y $ab>0$, entonces $\frac{1}{b}<\frac{1}{a}$. %K
    \item Si $a<1$ y $0<b$, entonces $ab<b$.
    \item Si $a<b$ demuestre que $a<\frac{a+b}{2}<b$.
    \item $a^2\geq 0$.
    \item Si $0 \leq a < \varepsilon$ para toda $\varepsilon > 0$, entonces $a=0$.
    \item Si $a \leq b + \varepsilon$ para toda $\varepsilon > 0$, entonces $a \leq b$.
\end{enumerate}

\subsubsection*{Demostración}

\begin{enumerate}[label=\alph*),font=\bfseries]

    %A

    \item Supongamos que $1 \notin \mathbb{R}^+$. Por (A7), (ii) de (O3) no se cumple. Si $-1 \in \mathbb{R}^+$, por (O2) se verifica que $-1 \cdot -1 \in \mathbb{R}^+$, lo cual por (h) de LE1 implica que $-(-1) \in \mathbb{R}^+$, pero esto contradice a (iii) de (O3). Por tanto, $1$ es un número real positivo.

    %B

    \item \begin{enumerate}[label=\roman*),font=\bfseries]
        \item Supongamos que $a \in \mathbb{R}^+$. Por A3 sabemos que $a=a+0$, y por (e) de LE1 sigue que $a=a-0$, entonces $a-0 \in \mathbb{R}^+$, lo que por definición implica que $a>0$.
        \item Supongamos que $a>0$. Por definición, $a-0 \in \mathbb{R}^+$, y por (e) de LE1 sigue que $a-0=a+0$, lo que por A3 implica que $a+0=a$. Así $a \in \mathbb{R}^+$.
        \end{enumerate}
    \qed

    %C

    \item Supongamos que $-1 \geq 0$
    \begin{enumerate}[label=\roman*),font=\bfseries]
        \item Si $-1=0$. Notemos que:
        \begin{align*}
        0 &= 1 - 1 && \text{Por A4}\\
        &= 1 + 0 && \text{Por hipótesis}\\
        &= 1 && \text{Por A3}\\
        \end{align*}
        Pero la igualdad anterior contradice a A7.
        \item Si $-1 > 0$, por (b) de LE3 tenemos que $-1 \in \mathbb{R}^+$, pero esto contradice a O3, ya que por (a) de LE3 sabemos que $1 \in \mathbb{R}^+$.
    \end{enumerate}
    Por tanto $-1<0$
    \qed

    %D

    \item Por definición $b-a \in \mathbb{R}^+$.
        \begin{enumerate}[label=\roman*), font=\bfseries]
        \item Si $c<d$, entonces $d-c \in \mathbb{R}^+$. Por (O1) se verifica que $(b-a)+(d-c) \in \mathbb{R}^+$. Notemos que:
        \begin{align*}
        b-a+d-c &= b+d-a-c && \text{Por A1}\\
        &= b+d+(-a)+(-c) && \text{Por notación}\\
        &= b+d+(-1)a+(-1)c && \text{Por (h) de LE1}\\
        &= b+d+ (-1) (a + c) && \text{Por A9}\\
        &= b+d + b-(a+c) \big) && \text{Por (h) de LE1}\\
        &= b+d - (a+c) && \text{Por notación}\\
        \end{align*}
        De este modo, $b+d-(a+c)\in \mathbb{R}^+$, es decir, $a+c<b+d$.
        \item Si $c=d$. Notemos que
        \begin{align*}
        b - a &= b -a + 0 && \text{Por A3}\\
        &= b-a+c-c && \text{Por A4}\\
        &= b+c-a-c && \text{Por A1}\\
        &= b+c+(-a)+(-c) && \text{Por notación}\\
        &= b+c+(-1)a+(-1)c && \text{Por (h) de LE1}\\
        &= b+c+(-1)(a+c) && \text{Por A9}\\
        &= b+c-(a+c) && \text{Por(h) de LE1}\\
        &= b+d-(a+c) && \text{Por hipótesis}\\
        \end{align*}
        De este modo, $b+d-(a+c)\in \mathbb{R}^+$, es decir, $a+c<b+d$.
        \end{enumerate}
        En cualquier caso, $a+c<b+d$. \qed

    %E

    \item Por definición $b-a \in \mathbb{R}^+$, y por (b) de LE3 $c \in \mathbb{R}^+$. Luego, por O2 se verifica que $c(b-a) \in \mathbb{R}^+$. Por A9 sigue que $c(b-a)=cb-ca$ y por A5 tenemos que $cb-ca=bc-ac$. De este modo, $bc-ac \in \mathbb{R}^+$, es decir, $ac<bc$.
    \qed

    %F

    \item Por definición $b-a \in \mathbb{R}^+$ y $0 - c \in \mathbb{R}^+$, por A3 sigue que $ -c \in \mathbb{R}^+$. Luego, por O2 $-c(b-a) \in \mathbb{R}^+$. Notemos que:
    \begin{align*}
        -c(b-a) &= -c \Bigl( b + (-a) \Bigr) && \text{Por notación}\\
        &= (-c) \cdot b + (-c) \cdot (-a) && \text{Por A9}\\
        &= (-c) \cdot b + c \cdot a && \text{Por (k) de LE1}\\
        &= -(c \cdot b) + c \cdot a && \text{Por (i) de LE1}\\
        &= ca -(cb) && \text{Por A1}\\
        &= ac - (bc) && \text{Por A5}\\
    \end{align*}
    Entonces $ac - bc \in \mathbb{R}^+$, es decir, $ac>bc$. \qed

    %G

    \item
    \begin{enumerate}[label=\roman*),font=\bfseries]
    \item Supongamos que $a \in \mathbb{R}^+$. Notemos que:
    \begin{align*}
        a &> 0 && \text{Por (b) de LE3}\\
        a \cdot (-1) &< 0 \cdot (-1) && \text{Por (c) y (f) de LE3}\\
        -a &< 0 && \text{Por (h) y (f) de LE1}
    \end{align*}
    \item Supongamos que $-a<0$. Notemos que:
    \begin{align*}
        -a \cdot (-1) &> 0 \cdot (-1) && \text{Por (b) y (f) de LE3}\\
        a &> 0 && \text{Por (l) y (f) de LE1}\\
    \end{align*}
    Entonces, $a \in \mathbb{R}^+$, por (b) de LE3.
    \end{enumerate}
    \qed

    %H

    \item Notemos que:
    \begin{enumerate}[label=\roman*),font=\bfseries]
        \item Si $a<b$, por (b) y (f) de LE3 tenemos que $a \cdot (-1) > b \cdot (-1)$, y por (h) de LE1 obtenemos que $-a > -b$.
        \item Si $-a > -b$, por por (b) y (f) de LE3 tenemos que $-a \cdot (-1) < -b \cdot (-1)$, y por (k) de LE1 obtenemos que $a<b$.
    \end{enumerate}
    \qed

    %I

    \item Sea $a>0$. Supongamos que $\frac{1}{a} \leq 0$. Notemos que:
    \begin{align*}
        \frac{1}{a} \cdot a &\leq 0 \cdot a && \text{Por (e) de LE3}\\
        1 &\leq 0 && \text{Por A8 y (f) de LE1}\\
    \end{align*}
    Pero por (a) de LE3 y (b) de LE3 tenemos que $1>0$. Por tanto, $\frac{1}{a} > 0$.
    \qed

    %J

    \item Por definición $b-a \in \mathbb{R}^+$ y $c-b \in \mathbb{R}^+$. Por O1 $(b-a) + (c-b) \in \mathbb{R}^+$. Notemos que:
    \begin{align*}
        (b-a) + (c-b) &= b - a + c -b && \text{Por notación}\\
        &= b-a -b+c && \text{Por A1}\\
        &= b-b -a+c && \text{Por A1}\\
        &= 0 - a +c && \text{Por A4}\\
        &= -a +c && \text{Por A3}\\
        &= c-a && \text{Por A1}\\
    \end{align*}
    Entonces $c-a \in \mathbb{R}^+$, es decir, $a<c$.
    \qed

    %K

    \item 
    \begin{enumerate}[label=\roman*),font=\bfseries]
        \item Si $a=0$ o $c=0$, por (g) de LE1 se verifica que $ac=0$. Luego, por (j) de LE3, se verifica que $0<b$ y $0<d$. Así, $ac<bd$.
        \item Si $a>0$ y $c>0$. Por hipótesis, $a<b$, y por (e) de LE3, sigue que $ac<bc$. También, tenemos que $c<d$, y por (e) de LE3, sigue que $bc<db$. Finalmente, por (j) de LE3, se verifica que $ac<bd$.
    \end{enumerate}
    \qed

    %L

    \item Notemos que:
    \begin{align*}
    a &< b && \text{Por hipótesis} \\
    a-a &< b-a && \text{Por (d) de LE3} \\
    0 &< b-a && \text{Por A4} \\
    0 \cdot \frac{1}{ab} &< (b-a) \cdot \frac{1}{ab} && \text{Por (h) y (e) de LE3}\\
    0 &< \frac{b-a}{ab} && \text{Por (f) de LE1 y (a) de LE2}\\
    0 &< \frac{1}{a} - \frac{1}{b} && \text{Por (c) de LE2}\\
    \frac{1}{b} &< \frac{1}{a} && \text{Por (d) de LE3}\\
    \end{align*}
    \qed

    %M

    \item Dado que $0<b$, por (b) de LE3 se cumple que $b \in \mathbb{R}^+$, y por definición $1-a \in \mathbb{R}^+$. Por O2 se verifica que $b(1-a) \in \mathbb{R}^+$, es decir, $b-ab \in \mathbb{R}^+$, lo cual implica que $ab<b$. \qed

    %N

    \item Por (a) de LE3 sabemos que $1 \in \mathbb{R}^+$, y por O1 se cumple que $1+1 \in \mathbb{R}^+$, es decir $2 \in \mathbb{R}^+$. Por (b) de LE3 se verifica que $0<2$ y por (i) de LE3 tenemos que $0<\frac{1}{2}$. Notemos que: \begin{align*}
        a &< b && \text{Por hipótesis} \\
        a + a &< b+a && \text{Por (d) de LE3} \\
        2a &< b+a && \text{Por definición} \\
        2a \cdot \frac{1}{2} &< (b+a) \cdot \frac{1}{2} && \text{Por (e) de LE3} \\
        \frac{2a}{2} &< \frac{b+a}{2} && \text{Por (a) de LE2} \\
        a &< \frac{b+a}{2} && \text{Por A8}
    \end{align*} \\ Similarmente,
    \begin{align*}
        a &< b && \text{Por hipótesis} \\
        a + b &< b+b && \text{Por (d) de LE3} \\
        a +b &< 2b && \text{Por definición} \\
        (a+b) \cdot \frac{1}{2} &< 2b \cdot \frac{1}{2} && \text{Por (e) de LE3} \\
        \frac{a+b}{2} &< \frac{2b}{2} && \text{Por (a) de LE2} \\
        \frac{a+b}{2} &< b && \text{Por A8}
    \end{align*}\\ Finalmente, por notación, $a < \frac{a+b}{2} < b$. \qed

    %O

    \item Si $0 \leq a$, $ 0\cdot a \leq a \cdot a$, osea, $0 \leq a^2$. Si $a<0$, $0\cdot a < a \cdot a$, osea, $0 \leq a^2$. En cualquier caso $a\geq0$.

    %P
    \item Supongamos que $0<a$, sigue que $0<\frac{a}{2}<a$. Elegimos $\varepsilon=\frac{a}{2}$, entonces $\varepsilon<a$, pero esto contradice nuestra hipótesis de que $a< \varepsilon$ para toda $\varepsilon>0$. Por tanto, $a=0$.\qed

    %Q
    \item Sean $a$ y $b$ números reales tales que $a \leq b + \varepsilon$, $\forall \varepsilon > 0$. Supongamos que $a > b$. Luego, $a-b>0$. Notemos que $(a-b) \cdot \frac{1}{2} > 0 \cdot \frac{1}{2}$, es decir $\frac{(a-b)}{2} > 0$. Sea $\varepsilon = \frac{(a-b)}{2}$, sigue que $a=2\varepsilon+b$. Además, $2\varepsilon > \varepsilon$, de donde obtenemos $2 \varepsilon + b > \varepsilon + b$. De este modo, $a > b+\varepsilon$, pero esto contradice nuestra hipótesis. Por tanto, $a \leq b$.\qed

\end{enumerate}

\textbf{Definición:} Sea $a$ un número real, definimos el valor absoluto de $a$, denotado por $|a|$ como sigue: 
    \[
    |a| = 
    \left \{
        \begin{aligned}
        a &,\ \text{si} \ a \geq 0\\
        -a &,\ \text{si} \ a < 0 \\
        \end{aligned}
    \right .
    \]

\textbf{Observación.} $|a|\geq 0, \ \forall a\in \R$.

\subsection*{Lista de Ejercicios 4 (LE4)}

Sean $a$, $b$, $c$ números reales, demuestre lo siguiente:

\begin{enumerate}[label=\alph*),font=\bfseries]
    \item $|a| \geq \pm a$. %A
    \item $|ab|=|a||b|$. %B
    \item $|a|=|-a|$. %C
    \item $|a+b|\leq |a|+|b|$. Desigualdad del triángulo. %D
    \item Si $b\neq 0$, entonces $\left| \frac{a}{b} \right| = \frac{|a|}{|b|}$. %E
    \item $|a|<b$ si y solo si $-c<b<c$. %F
    \item $ \big| |a|-|b| \big| \leq |a-b|$ %G
    \item $|a|^2=a^2$.
\end{enumerate}

\subsubsection*{Demostración}

\begin{enumerate}[label=\alph*),font=\bfseries]

    %A

    \item
        \begin{enumerate}[label=\roman*),font=\bfseries]
            \item Si $a \geq 0$, entonces $|a|=a$, así, $|a| \geq a$. Luego, $-a \leq 0$, de donde sigue que $a \geq -a$. Finalmente, $|a| \geq -a$.
            \item Si $a<0$, entonces $|a|=-a$, así, $|a| \geq -a$. Luego, $-a>0$, de donde sigue que $-a>a$. Finalmente, $|a| \geq a$.
        \end{enumerate}
        En cualquier caso, $|a| \geq \pm a$.

    %B

    \item 
        \begin{enumerate}[label=\roman*),font=\bfseries]
            \item Si $a>0$ y $b>0$, entonces $|a|=a$ y $|b|=b$. Luego, $ab>0$ por lo que $|ab|=ab$. De este modo, $|ab| =|a||b|$.
            \item Si $a>0$ y $b<0$, entonces $|a|=a$ y $|b|=-b$. Luego, $ab<0$ por lo que $|ab|=-ab$. De este modo, $|  ab|=|a||b|$.
            \item Si $a<0$ y $b<0$, entonces $|a|=-a$ y $|b|=-b$. Luego, $ab>0$ por lo que $|ab|=ab$. De este modo, $|  ab|=|a||b|$.
        \end{enumerate}

    %C

    \item 
        \begin{enumerate}[label=\roman*),font=\bfseries]
            \item Si $a \geq 0$, entonces $|a|=a$. Luego, $-a \leq 0$. Si $-a<0$, $|-a|=a$ y si $-a=0$, $|-a|=a$. De este modo, $|a|=|-a|$.
            \item Si $a<0$, entonces $|a|=-a$. Luego, $-a>0$ por lo que $|-a|=-a$. De este modo, $|a|=|-a|$.
        \end{enumerate}

    %D
    
    \item 
        \begin{enumerate}[label=\roman*),font=\bfseries]
            \item Si $0 \leq a+b$, entonces $|a+b|=a+b$. Además, $a \leq |a|$ y $b \leq |b|$. Luego, $a+b \leq |a|+|b|$. Así, $|a+b| \leq |a|+|b|$.
            \item Si $0 > a+b$, entonces $|a+b|=-a-b$. Además, $-a \leq |a|$ y $-b \leq |b|$. Luego, $-a-b \leq |a|+|b|$. Así, $|a+b| \leq |a|+|b|$.
        \end{enumerate}
    
    %E

    \item 
        \begin{enumerate}[label=\roman*),font=\bfseries]
            \item Si $a \geq 0$ y $b>0$, entonces $|a|=a$ y $|b|=b$. Además, $\frac{1}{b} >0$, de donde sigue que $\frac{a}{b} \geq 0$ por lo que $\big| \frac{a}{b} \big| = \frac{a}{b}$. De este modo, $ \big| \frac{a}{b} \big| = \frac{|a|}{|b|}$.
            \item Si $a \geq 0$ y $b<0$, entonces $|a|=a$ y $|b|=-b$. Además, $\frac{1}{b} <0$, de donde sigue que $\frac{a}{b} \leq 0$, por lo que $\big| \frac{a}{b} \big| =- \frac{a}{b}$. De este modo, $ \big| \frac{a}{b} \big| = \frac{|a|}{|b|}$.
            \item Si $a<0$ y $b>0$, entonces $|a|=-a$ y $|b|=b$. Además, $\frac{1}{b} >0$, de donde sigue que $\frac{a}{b} < 0$, por lo que $\big| \frac{a}{b} \big| =- \frac{a}{b}$. De este modo, $ \big| \frac{a}{b} \big| = \frac{|a|}{|b|}$.
            \item Si $a<0$ y $b<0$, entonces $|a|=-a$ y $|b|=-b$. Además, $\frac{1}{b} <0$, de donde sigue que $\frac{a}{b} > 0$ por lo que $\big| \frac{a}{b} \big| = \frac{a}{b}$. De este modo, $ \big| \frac{a}{b} \big| = \frac{|a|}{|b|}$.
        \end{enumerate}

    %F
        
    \item
        \begin{enumerate}[label=\roman*),font=\bfseries]
            \item Supongamos que $|b|<c$. Por (a) de LE4, $ \pm b \leq |b|$, de donde sigue que $-b<c$ y $b<c$. Luego, $-c<b$. De este modo, $-c<b<c$.
            \item Supongamos que $-c<b<c$. Luego,
                \begin{enumerate}[label=\arabic*),font=\bfseries]
                    \item Si $b \geq 0$, entonces $|b|=b$. Por lo que $|b|<c$.
                    \item Si $b < 0$, entonces $|b|=-b$. Por hipótesis, $-c<b$, por lo que $-b<c$. Así $|b|<c$.
                \end{enumerate}
        \end{enumerate}
    
    %G

    
    \item Por la desigualdad del triángulo,
        \begin{align*}
            |(a-b)+b| &\leq |a-b|+|b| \\
            |a| &\leq |a-b|+|b| \\
            |a|-|b| &\leq |a-b| && \text{(1)}
        \end{align*} \\
        Similarmente, 
        \begin{align*}
            |(b-a)+a| &\leq |b-a|+|a| \\
            |b| &\leq |b-a|+|a| \\
            |b|-|a| &\leq |b-a| \\
            -|b-a| &\leq |a|-|b| && \text{(2)}
        \end{align*} \\
        Luego, aplicando (f) de LE4 en (1) y (2), $\big| |a| - |b| \big| \leq |a-b|$.

    %H

    \item Por (o) de LE3, $a^2\geq 0$, por lo que \begin{align*}
        a^2 &= |a^2|\\
        &= |a\cdot a|\\
        &= |a| \cdot |a| && \text{Por (b) de LE4}\\
        &= |a|^2
    \end{align*} \qed
\end{enumerate} 

%\pagebreak
%VECINDAD

\textbf{Definición.} Sea $a \in \R$ y $\varepsilon>0$. El vecindario-$\varepsilon$ de $a$ es el conjunto $V_\varepsilon(a):=\{ x\in \R: |x-a|<\varepsilon\}$.

\subsection*{Lista de Ejercicios 5 (LE5)}

Sean $a,b \in \R$. Demuestre lo siguiente:

\begin{enumerate}[label=\alph*)]
    \item Si $x\in V_\varepsilon(a)$ para toda $\varepsilon>0$, entonces $x=a$.
    \item Sea $U:=\{x: 0<x<1\}$. Si $a\in U$, sea $\varepsilon$ el menor de los números $a$ y $1-a$. Demuestre que $V_\varepsilon(a) \subseteq U$.
    \item Demuestre que si $a\neq b$, entonces existen $U_\varepsilon(a)$ y $V_\varepsilon(b)$ tales que $U\cap V =\emptyset$.
\end{enumerate}

\textbf{Demostración.}

\begin{enumerate}[label=\alph*)]
    \item Si $x\in V_\varepsilon(a)$ tenemos que $|x-a|<\varepsilon$. Además, $0\leq |x-a|$, por definición. Así, $0\leq |x-a|<\varepsilon$. Como esta desigualdad se cumple para toda $\varepsilon>0$, por (p) de LE3, sigue que $|x-a|=0$. De este modo, $|x-a|=x-a$ con $x-a=0$. Por tanto, $x=a$. \qed
    
    \item \begin{enumerate}[label=\roman*)]
        \item Si $a>1-a$, tenememos $\varepsilon=1-a$. Sea $y\in V_\varepsilon(a)$, entonces $|y-a|<1-a$. De (f) de LE4 sigue que $a-1<y-a<1-a$ (*). Tomando el lado derecho de (*) obtenemos $y<1$. Luego, de la hipótesis sigue que $2a>1$, osea $2a-1>0$. Del lado izquierdo de la desigualdad (*), tenemos $2a-1<y$, por lo que $0<y$.
        \item Si $1-a>a$, tenemos $\varepsilon=a$. Sea $y\in V_\varepsilon(a)$, entonces $|y-a|<a$. De (f) de LE4 sigue que $-a<y-a<a$. Sumando $a$ en esta desigualdad obtenemos $0<y<2a$. Luego, de la hipótesis sigue que $1>2a$, entonces $0<y<1$.\end{enumerate}
        En cualquier caso, $0<y<1$, lo que implica que $V_\varepsilon(a) \subseteq U$.
        \qed

    \item Supongamos que para toda $U_\varepsilon(a)$ y $V_\varepsilon(b)$ se cumple que $U_\varepsilon(a) \cap V_\varepsilon(b) \neq \emptyset$. Entonces, existe $x$ tal que $x\in U_\varepsilon(a)$ y $x\in V_\varepsilon(b)$. Como en ambas vecindades tenemos $\epsilon>0$ arbitraria, por (a) de LE5, sigue que $x=a$ y $x=b$, pero esto contradice el supuesto de que $a\neq b$. Por tanto, deben existir $U_\varepsilon(a)$ y $V_\varepsilon(b)$ tales que $U\cap V =\emptyset$. \qed
    \end{enumerate}

\textbf{Definición:} Sea $A$ un subconjunto del conjunto de los números reales, decimos que $A$ es un conjunto inductivo si se cumplen las siguientes condiciones:
    \begin{enumerate}
        \item $1 \in A$.
        \item Si $n \in A$ entonces se verifica que $n+1 \in A$.
    \end{enumerate}

\subsection*{Lista de Ejercicios 6 (LE6)}

\begin{enumerate}[label=\arabic*)]
    \item ¿El conjunto de los números reales es un conjunto inductivo?
    \item ¿$\R^+$ es un conjunto inductivo?
    \item Sea $A\coloneqq \{B \subseteq B: \text{B es un conjunto inductivo}\}$. Demuestre que $A\neq \emptyset$ y que $C=\bigcap B$ es un conjunto inductivo.
\end{enumerate}

\textbf{Respuesta}

\begin{enumerate}[label=\arabic*)]
    \item Sí, ya que $1 \in \R$, y si $n$ es un número real, $n+1 \in \R$ por la cerradura de la suma en $\R$.
    \item Sí, pues $1\in \R^+$ y y si $n$ es un número real positivo, $n+1 \in \R^+$ por el axioma de orden 1.
    \item Claramente $A \neq \emptyset$, pues $\R, \R^+ \subseteq A$. \\[5pt]Luego, por hipótesis, $\forall B \in A$ tenemos que $B\subseteq \R $ por lo que $C\subseteq \R$. Además, $\forall B\in A$, se verifica que $1\in B$. Consecuentemente, $1\in C$. Por otra parte, si $n\in B$ para todo $B\in A$, tendremos que $n+1\in B$, por lo que $n+1 \in C$. Por tanto, $C$ es un conjunto inductivo.
\end{enumerate}

\textbf{Definición.} Al conjunto $C$ de (3) de LE6 lo llamaremos conjunto de los números naturales y lo denotaremos con el símbolo $\N$.

\subsection*{Lista de ejercicios 7 (LE7)}

Demuestre lo siguiente:

\begin{enumerate}[label=\alph*)]
    \item La suma de números naturales es un número natural.
    \item La multiplicación de números naturales es un número natural.
    \item $n\geq 1, \forall n\in \N$.
    \item $0<n^{-1}\leq 1, \forall n\in \N$.
    \item $\forall n\in \N$ con $n>1$ se verifica que $n-1\in \N$.
    \item Sean $m$ y $n$ números naturales tales que $m>n$, demuestre que $m-n\in\N$.
    \item Sea $x\in \R^+$. Si $n\in \N$ y $x+n\in \N$, desmuestre que $x\in \N$.
    \item Sea $x\in \R$, si $n\in \N$ y $n-1<x<n$, demuestre que $x$ no es un número natural.
\end{enumerate}

\textbf{Demostración.}

\begin{enumerate}[label=\alph*)]
    \item Sea $m\in \N$ arbitrario pero fijo. Definimos $A=\{ n\in \N : m+n \in \N \}$. Por definición, $1\in \N$ y $m+1\in \N$, entonces $1\in A$, es decir, $A\neq \emptyset$. \\[5pt] Por otra parte, si $n\in A$ debe ser el caso que $n\in \N$ y $m+n\in \N$. Como $\N$ es un conjunto inductivo, $n+1 \in \N$ y $(m+n)+1 \in \N$, luego, por la asociatividad de la suma, $m+(n+1)\in \N$. Por la condición de $A$, se cumple que $n+1\in A$, por lo que $A$ es un conjunto inductivo. De esto se concluye que $\N\subseteq A$ y como $A\subseteq \N$, $A=\N$. En otras palabras, la suma de números naturales es un número natural. \qed
    
    \item Sea $m\in \N$ arbitrario pero fijo. Definimos $A=\{n\in \N: m\cdot n \in \N\}$. Por definición, $1 \in \N$. Adenás, $m\cdot 1 \in \N$, entonces $1 \in A$, es decir $A \neq \emptyset$.\\[5pt] 
    Luego, si $n \in A$ debe ser el caso que $n\in \N$ y $m \cdot n \in \N$. Por (a) de LE7 se verifica que $(m \cdot n) + m \in \N$. Notemos que $(m \cdot n) + m=m \cdot (n+1)$, osea, $m \cdot (n+1) \in \N$. Como $\N$ es un conjunto inductivo, tenemos que $n+1\in \N$. De este modo, $n+1\in A$. Lo que implica que $A$ es un conjunto inductivo. De esto se concluye que $\N \subseteq A$ y como $A\subseteq \N$, $A=\N$. En otras palabras, la multiplicación de números naturales es un número natural. \qed

    \item Sea $A\coloneqq \{n\in \N: n\geq 1\}$. Como $1\in \N$ y $1\geq 1$, tenemos que $1\in A$.\\[5pt]
    Si $n\in A$ debe ser el caso que $n\in \N$ y $1\leq n$. Además, por (a) de LE7, $n+1\in \N$. Luego, notemos que $0 \leq 1$ de donde sigue que $n \leq n+1$. Por transitividad, $1\leq n+1$, por lo que $n+1\in A$, lo que implica que $A$ es un conjunto inductivo, es decir, $\N\subseteq A$ y como $A\subseteq \N$, $A=N$. En otras palabras, $n\geq 1, \forall n\in\N$. \qed

    \item Por (d) de LE7, $n\geq 1, \forall n\in \N$. Si $n=1$, tenemos que $n^{-1}=1>0$. Si $n>1$, tenemos que $n>0$, por lo que $n^{-1}>0$. En cualquier caso, $1\geq n^{-1}>0, \forall n\in \N$.

    \item Sea $A \coloneqq \set{n\in \N | n>1, n-1\in \N}$. Si $n\in A$ debe ser porque $n>1$ y $n-1\in \N$. Como $n\in \N$ y $\N$ es un conjunto inductivo, se verifica $n+1\in\N$. Notemos que \begin{align*}
        (n+1)-1 &= n+(1-1) \\
        &= n+ 0\\
        &= n
    \end{align*}\\
    Entonces, $(n+1)-1\in \N$. También, $n>1$ implica que $n>0$ y $n+1>1$, por lo que $n+1\in A$. De este modo, $A$ es un conjunto inductivo, con lo que $\N \subseteq A$, y como $A\subseteq \N$, $A=\N$. Por tanto $\forall n\in \N$ con $n>1$ se verifica que $n-1\in \N$. \qed

    \item Sea $A \coloneqq \set{n\in \N| n<m, m-n\in\N \ \text{con} \ m\in\N}$. Por definición, $1\in \N$ y $1+1\in \N$. Por (a) y (b) de LE3, $1>0$, de donde sigue que $1+1>1$. Por (e) de LE7, se verifica que $(1+1)-1\in \N$, por lo que $1\in A$. \\[5pt] Si $n \in A$ debe ser porque $m-n\in \N$ y $m>n$, de donde obtenemos $m+1>n+1$. Como $m,n\in \N$ y $\N$ es un conjunto inductivo, $n+1\in \N$ y $m+1 \in \N$. Notemos que $m+1-(n+1)=m-n$, por lo que $n+1\in A$. De este modo, $A$ es un conjunto inductivo, con lo que $\N \subseteq A$, y como $A\subseteq \N$, $A=\N$. \qed 
    
    %Para algún $m\in \N$ tal que $m>1$, por (e) de LE7, se verifica que $m-1\in \N$, por lo que $1\in A$. \\[5pt] \textbf{Nota}: Se abusa de la notación y no debe entenderse que $m$ es fija, sino que depende del número natural con que se relaciona.   
    
    \item Por (b) de LE3, $x>0$, por lo que $x+n>n$. Por hipótesis, $x+n, n\in \N$, y por (f) de LE7 $(x+n)-n \in \N$, osea, $x\in \N$. \qed
    
    \item Supongamos que $x\in \N$. Por hipótesis tenemos que $x<n$ y $x>n-1$. Notemos que \begin{align*}
        x &< n\\
        x -n &< n-n\\
        x-n &< 0\\
        x-n +1 &< 1
    \end{align*}\\
    Del mismo modo, \begin{align*}
        n-1 &< x\\
        n-1-(n-1) &< x - (n-1)\\
        0 &< x-n+1\\
        n &< x+1
    \end{align*}\\
    Como $\N$ es un conjunto inductivo, $x+1\in \N$, y como $x+1>n$, con $n\in \N$, por (e) de LE7, $x+1-n \in \N$, y por (c) de LE7, $x+1-n\geq 1$. Pero tenemos que $x-n+1<1$, osea $1\leq x+1-n<1$, lo cual es una contradicción. Por tanto, $x$ no es un número natural. \qed
\end{enumerate}

\pagebreak

\textbf{Definición.} Sea $E$ un subconjunto no vacío de $\R$, decimos que $E$ está acotado: \begin{itemize}
    \item Superiormente si existe un número real $m$ tal que $b \leq m, \forall b\in E$. En este caso decimos que $E$ es cota superior de $E$.
    \item Inferiormente si existe un número real $l$ tal que $l \leq b, \forall b\in E$. En este caso, decimos que $l$ es cota inferior de $E$.
    \item Si existe un número real $m$ tal que $|b|\leq m,\forall b \in E$. En este caso decimos que $m$ es una cota de $E$.
\end{itemize}

\textbf{Definición.} Sea $A$ un subconjunto no vacío del conjunto de los números reales, acotado superiormente, decimos que un número real $M$ es supremo de $A$ si $M$ satisface las siguientes condiciones: \begin{itemize}
    \item $M$ es cota superior de $A$.
    \item Si $K$ es una cota superior de $A$, entonces $M\leq K$, es decir, $M$ es la cota superior más pequeña de $A$.
\end{itemize}

En este caso escribimos $M=\sup{A}$.

\textbf{Definición}. Sea $A$ un subconjunto no vacío del conjunto de los números reales, acotado inferiormente, decimos que un número real $L$ es ínfimo de $A$ si $L$ satisface las siguientes condiciones: \begin{itemize}
    \item $L$ es cota inferior de $A$.
    \item Si $K$ es una cota inferior de $A$, entonces $K\leq L$, es decir, $L$ es la cota inferior más grande de $A$.
\end{itemize}

En este caso escribimos $M=\inf{A}$.

\subsection*{Lista de ejercicios 8 (LE8)}

Falso o verdadero: \begin{enumerate}[label=\arabic*.]
    \item Si $E$ es un subconjunto de $\R$ acotado superiormente, entonces $E$ es un conjunto acotado.
    \item Si $E$ es un subconjunto acotado de $\R$, entonces $E$ está acotado superiormente e Inferiormente.
\end{enumerate}

Demuestre lo siguiente:

\begin{enumerate}[label=\arabic*.]\setcounter{enumi}{2}
    \item Sea $A$ un subconjunto no vacío de $\R$, si $A$ tiene supremo, este es único.
    \item Sea $A$ un subconjunto no vacío de $\R$, si $A$ tiene ínfimo, este es único.
    \item Una cota superior $M$ de un conjunto no vacío $S$ de $\R$ es el supremo de $S$ si y solo si para toda $\varepsilon>0$ existe una $s_\varepsilon \in S$ tal que $M-\varepsilon<s_\varepsilon$.
\end{enumerate}

\subsubsection*{Respuesta}

\begin{enumerate}[label=\arabic*.]
    \item Falso. Consideremos el conjunto $\R\backslash \R^+$, el cual es un subconjunto de $\R$, y es no vacío, pues $-1\in \R\backslash \R^+$. Además, $b\leq 0, \forall b\in \R\backslash\R^+$, por lo que el conjunto está acotado superiormente. Supongamos que el conjunto propuesto está acotado. Es decir, suponemos que $\exists m$ tal que $|b|\leq m, \forall b\in \R\backslash \R^+$. Por (f) de LE4, $-m \leq b$ y, por transitividad, $-m\leq 0$, de donde sigue que $-m-1\leq -1$, pero $-1<0$, entonces $-m-1<0$, lo que implica que $-m-1\in \R\backslash \R^+$, por lo que $|-m-1|\leq m$. Luego, notemos que $|-m-1|=-(-m-1)$, es decir, tenemos que $m+1\leq m$, pero de esto se concluye que $1\leq 0$, lo cual es una contradicción. Por tanto, aunque $\R\backslash \R^+$ está acotado superiormente, no está acotado.
    \item Verdadero. Sea $E$ un subconjunto no vacío de $\R$. Si $E$ está acotado, entonces $\exists m$ tal que $|b|\leq m,\forall b \in E$. Por (f) de LE4, $-m\leq b \leq m$, por lo que el conjunto está acotado superiormente e inferiormente.
\end{enumerate}

\subsubsection*{Demostración}

\begin{enumerate}[label=\arabic*.]\setcounter{enumi}{2}
    \item Supongamos que $s_1$ y $s_2$ son supremos de $A$. Como $s_1$ es una cota superior de $A$ y $s_2$ es elemento supremo, entonces $s_2\leq s_1$. Similarmente, $s_1\leq s_2$. Por tanto, $s_1=s_2$. \qed
    \item Supongamos que $m_1$ y $m_2$ son ínfimos de $A$. Como $m_1$ es una cota superior de $A$ y $m_2$ es elemento ínfimo, entonces $m_1\leq m_2$. Similarmente, $m_2\leq m_1$. Por tanto, $m_1=m_2$. \qed
    \item \begin{enumerate}[label=\roman*)]
        \item Sea $M$ una cota superior de $S$ tal que $\forall \epsilon>0, \exists s_{\epsilon}$ tal que $M-\epsilon<s_{\epsilon}$. Si $M$ no es el supremo de $S$, tendríamos que $\exists V$ tal que $s_@a \leq V < M$. Elegimos $\epsilon = M-V$, con lo que $V<s_{\epsilon}$, lo que contradice nuestra hipótesis. Por tanto, $M$ es el supremo de $S$.
        \item Sea $M$ el supremo de $S$ y $\epsilon>0$. Como $M<M+\epsilon$, entonces $M-\epsilon$ no es una cota superior de $S$, por lo que $\exists s_\epsilon$ tal que $s_\epsilon>M-\epsilon$.
        \end{enumerate}
        \qed
\end{enumerate}

\subsection*{Principio del buen orden}

Todo subconjunto no vacío del conjunto de los números naturales tiene elemento mínimo. Esto significa que si $A\subseteq \N$ y $A \neq \emptyset$, entonces existe un elemento $c\in A$ tal que $c\leq a, \forall a\in A$.

\textbf{Observación:}

Sabemos —por (c) de LE7— que cualquier subconjunto no vacío de $\N$ está acotado inferiormente. El principio del buen orden nos garantiza que cualquier subconjunto no vacío de $\N$ contiene una de sus cotas inferiores, a la que llamamos elemento mínimo.

Notemos que si suponemos la existencia de un subconjunto no vacío de $\N$ tal que ninguna de sus cotas inferiores esté contenida en el conjunto, estaríamos negando el principio del buen orden. Es así cómo procedemos a probar el teorema.

\pagebreak

\textbf{Demostración:}

Sea $A\subseteq \N$ con $A\neq \emptyset$. Supongamos que $A$ no contiene ninguna de sus cotas inferiores, es decir, supongamos que si $c\leq a, \forall a\in A$, entonces $c\notin A$.

Definimos el conjunto $L\coloneqq \set{n\in \N: n\leq a, \forall a\in A}$. Es claro que $1\in L$. Veamos que si $n\in L$, tendríamos que $n\leq a, \for a\in A$. Luego, si $n+1\notin L$, entonces $\exists a_0\in A$ tal que $n+1>a_0$, por lo que $n\leq a_0<n+1$, y —por (h) de LE7— no puede ser el caso que $n<a_0$, de donde sigue que $n=a_0$, pero esto contradice nuestro supuesto inicial, entonces, debe ser el caso que $n+1\in L$. Consecuentemente, $L$ es un conjunto inductivo, y —por definición— $\N\subseteq L$ y $L\subseteq \N$, lo que implica que $L=\N$.

Finalmente, notemos que $A$ y $L$ son disjuntos, y dado que $A\subseteq \N$ y $L=\N$, sigue que $A=\emptyset$, pero esto es una contradicción. Por tanto, si $A\subseteq \N$ con $A\neq \emptyset$, entonces $\exists c\in A$ tal que $c\leq a, \forall a\in A$. \qed

%\textbf{\textit{Teorema.}} Si $A\subseteq \N$ y $A\neq \emptyset$ y $A$ está acotado superiormente, entonces $A$ tiene elemento máximo, esto es existe un elemento $c\in A$ tal que $a\leq c, \for a\in A$.
%
%\textbf{Demostración:}
%
%Sea $A\subseteq\N$ con $A\neq \emptyset$ y $A$ acotado superiormente. Supongamos que si $c\geq a, \for a\in A$, entonces $c\notin A$.
%
%Definimos el conjunto $-A\defined \set{-a: a\in A}$. Como $A$ está acotado superiormente, $\exists c\in \R$ tal que $c\geq a, \forall a\in A$. Notemos que $-a\geq -c, \forall -a\in -A$, lo que implica que $-A$ está acotado inferiormente, y por esto, $A$ tiene elemento mínimo. Sea $m$ el elemento mínimo de $-A$. Veamos que $m\leq -a, \forall -a\in -A$ de donde sigue que $a\leq -m, \forall a\in A$, con $-m\in A$ pero esto contradice nuestro supuesto inicial. Por tanto, $A$ tiene elemento máximo. \qed
%
\subsection*{Axioma del supremo}

Todo subconjunto no vacío del conjunto de los números reales que sea acotado superiormente tiene supremo.

\textbf{\textit{Teorema.}} El conjunto de los números naturales no está acotado superiormente.

\textbf{Demostración:}

Supongamos que el conjunto de los números naturales está acotado superiormente. Entonces existe un número real $M$ tal que $n\leq M, \forall n\in \N$. Como el conjunto de los números naturales es no vacío, entonces, por el axioma del supremo, $\N$ tiene supremo.

Sea $L\coloneqq \sup{(\N)}$. Como $L-1$ no es cota superior de $\N$, ya que $L>L-1$ y $L$ es la cota superior más pequeña, existe un núero natural $n_0$ tal que $n_0>L-1$, lo cual implica que $n_0+1<L$, pero esto contradice la hipótesis	de que $L$ es supremo de $\N$. Por tanto, el conjunto de los números naturales no está acotado superiormente. \qed

\textbf{\textit{Teorema.}} Si $A\subseteq \R, A\neq \emptyset$ y $A$ está acotado inferiormente, entonces $A$ tiene ínfimo.

\textbf{Demostración:}

Sea $A\subseteq \R, A\neq \emptyset$ y $A$ está acotado inferiormente. El conjunto $-A \coloneqq \set{-a: a\in A}$ está acotado superiormente y, por el axioma del supremo, $-A$ tiene supremo. Sea $M\coloneqq \sup{(A)}$, entonces $M\geq -a, \forall -a\in -A$. Notemos que $-M\leq a, \forall a\in A$, esto es $-M$ es el ínfimo de $A$. \qed

\subsection*{Propiedad Arquimediana del conjunto de los números reales}

Para cada número real $x$ existe un número natural $n$ tal que $x<n$.

\textbf{Demostración:}

Supongamos que existe $x\in \R$ tal que $n\leq x, \forall n\in \N$. Notemos que $x$ es una cota superior de $\N$, pero esto contradice el teorema que establece que el conjunto de los números naturales no está acotado superiormente. Por tanto, se satisface la propiedad arquimediana del conjunto de los números reales. \qed

\textbf{Definción.} \begin{itemize}
    \item Al conjunto $\N \cup {0} \cup {-n: n\in \N}$ lo llamaremos conjunto de los números enteros y lo representaremos con el símbolo $\Z$.
    \item Al conjunto ${-n: n\in \N}$ lo llamaremos conjunto de los números enteros negativos y lo representaremos con el símbolo $\Z^-$.
    \item Al conjunto $\N$ también lo llamaremos conjunto de los números enteros positivos y lo representaremos con el símbolo $\Z^+$.
\end{itemize}

\textbf{Observación.} Los conjuntos $\N$, ${0}$, ${-n: n\in \N}$ son disjuntos por pares.

\subsection*{Lista de Ejercicios 9 (LE9)}

\begin{enumerate}[label=\alph*)]
    \item Si $S \coloneqq \set{\frac{1}{n}: n\in \N}$, entonces $\inf{S=0}$.
    \item Si $t>0$, entonces $\exists n\in \N$ tal que $0<\frac{1}{n}<t$.
    \item Si $y>0$, entonces $\exists n\in \N$ tal que $n-1\leq y< n$.
    \item Sea $x\in \R$, demuestre que $\exists! \, n\in \Z$ tal que $n\leq x<n+1$.
\end{enumerate}

\textbf{Demostración:}

\begin{enumerate}[label=\alph*)]
    \item Por (d) de LE7, $S$ está acotado inferiormente por $0$; de esto sigue que $S$ tiene ínfimo. Sea $w\coloneqq \inf{S}$. Por definición, $\frac{1}{n}\geq w\geq 0, n\in \N$. Supongamos que $w>0$. Por la propiedad arquimediana $\exists n_0$ tal que $\frac{1}{w} < n_0$, de donde sigue que $w<\frac{1}{n_0}$ con $\frac{1}{n_0} \in S$, lo cual es una contradicción. Por tanto, $w=0$. \qed
    
    \item Por la propiedad arquimediana $\exists n$ tal que $\frac{1}{t}<n$. Como $n$ y $t$ son mayores que $0$, sigue que $0<\frac{1}{n}<t$. \qed
    
    \item Por la propiedad arquimediana, el conjunto $E\coloneqq \set{m\in \N: y<m}$ es no vacío. Además, por el principio del buen orden, $\exists n\in E$ tal que $n\leq m, \for m\in E$. Notemos que $n-1<n$, por lo que $n-1\notin E$, lo que implica que $n-1\leq y<n$. \qed
    
    \item Definimos el conjunto $A\defined \set{n\in \Z: x<n}$. Por la propiedad arquimediana $\exists n_0 \in \N$ tal que $x<n_0$, así $n_0\in A$, por lo que $A\neq \emptyset$. Sabemos también que $A$ está acotado inferiormente, de manera que $A$ tiene elemento mínimo. Sea $n$ el elemento mínimo de $A$. Notemos que $n-1<n$, de donde sigue que $n-1\leq x<n$. Luego, $n-1\in \Z$, al que definimos como $m=n-1$, por lo que $m\leq x<m+1$.
    
    Finalmente, supongamos que $\exists m, n\in \Z$ tales que $m\leq x<m+1$ y $n\leq x<n+1$. Si $m\neq n$, sin pérdida de generalidad, $m>n$. Por ello, \begin{align*}
        n < m &\leq x<n+1 \\
        n < m &<n+1 \\
        0 < m-n &<1
    \end{align*}\\    
    Lo que contradice la cerradura de la suma en $\Z$. Por tanto, $m=n$, es decir, el número entero que satisface $n\leq x<n+1$ es único. \qed
    
    %\textbf{Demostración alternativa:}
%
    %Definimos el conjunto $A\defined \set{n\in \Z: n\leq x}$. Por la propiedad arquimediana $\exists n_0\in \N$ tal que $n_0>x$. Observemos que \begin{enumerate}[label=\roman*)]
    %\item Si $x\geq 0$, $-x\leq 0$ y $-x<n_0$, de donde sigue que $-n_0<x$, por lo que $-n_0\in A$.
    %\item Si $x<0$, $x<n_0$, de donde sigue que $-n_0\leq x$, por lo que $-n_0\in A$.
    %\end{enumerate} Consecuentemente, $A$ es no vacío. También sabemos que $A$ está acotado superiormente, por el axioma del supremo, $A$ tiene supremo. Sea $m\defined \sup{A}$. Por definición, $m\leq x$. Notemos que $m+1>m$. Luego, si $m+1\leq x$ tendríamos que $m+1\in A$ pero como $m$ es el supremo de $A$ seguiría que $m \geq m+1$, lo cual es una contradicción, entonces debe ser el caso que $m\leq x<m+1$.\qed
\end{enumerate}

\subsection*{Lista de Ejercicios \# (LE\#)}

Sean $a$ y $b$ números reales, demuestre lo siguiente:

\begin{enumerate}[label=\alph*),font=\bfseries]
    \item $0 \leq a^{2n} \, \forall n\in \N$.
    \item Si $0\leq a$, entonces $ 0 \leq a^n, \, \forall n\in \N$.
    \item Si $0 \leq a <b$, entonces $a^n < b^n, \, \forall n\in \N$.
    \item Si $0 \leq a <b$, entonces $a^n \leq ab^n < b^n \, \forall n\in \N$.
    \item Si $0<a<1$, entonces $a^n<a \, \forall n\in \N$.
    \item Si $1<a$, entonces $a<a^n \, \forall n\in \N$.
\end{enumerate}

\subsubsection*{Demostración}

\begin{enumerate}[label=\alph*),font=\bfseries]

    %A
    \item Pendiente

    %B
    \item Por inducción matemática. 

    \begin{enumerate}[label=\roman*)]
        \item Verificamos que se cumple para $n=1$. \begin{align*}
        0 &\leq a^1 \\
        0 &\leq a
        \end{align*}
        \item Suponemos que se cumple para $n=k$, para algún $k \in \N$. Es decir,  suponemos que \[0 \leq a^k\]
        \item Probaremos a partir de (ii) que $0 \leq a^{k+1}$. En efecto, por hipótesis de     inducción \begin{align*}
        0 &\leq a^k \\
        0 \cdot a &\leq a^k \cdot a \\
        0 &\leq a^{k+1}
        \end{align*}
    \end{enumerate}

    %C
    \item Por inducción matemática.
    \begin{enumerate}[label=\roman*)]
        \item Verificamos que se cumple para $n=1$. \begin{align*}
        a^1 &< b^1 \\
        a &< b
        \end{align*}
        \item Suponemos que se cumple para $n=k$, para algún $k\in \N$. Es decir, suponemos que \[a^k < b^k\]
        \item Probaremos, a partir de (ii) que $a^{k+1} < b^{k+1}$. En efecto, por (c) de LE5, garantizamos que $0 \leq a^k$, lo que nos permite, por (a) de LE5, afirmar que
        \begin{align*}
        a^k \cdot a &< b^k \cdot b \\
        a^{k+1} &< b^{k+1}
        \end{align*}
    \end{enumerate}

    %D
    \item Tenemos que $a<b$, como $0\leq a<b$, sigue que $0<b$, entonces $a\cdot b < b\cdot b$, osea $ab<b^2$. Luego, $a \cdot a \leq ab$. Finalmente, $a^2\leq ab < b^2$.
    
    %E
    \item Pendiente
    
    %F
    \item Pendiente

\end{enumerate}

\textbf{Definición:} Una secuencia (sucesión) es una función %$X: n\in \N \mapsto x_n \in \R$
\begin{align*}
    X: \ & \N \to \R \\
    \ &  n \mapsto x_n 
\end{align*}

Llamamos a $x_n$ el n-ésimo término. Otras etiquetas para la secuencia son $(x_n)$, $(x_n:n\in \N)$, donde esta última denota orden y se diferencia del rango de la función $\{x_n:n\in \N\}\subseteq \R$.

\textbf{Definición:} Decimos que una secuencia $X$ converge a $\ell \in \R$, o que la secuencia es convergente, si $\exists n_0\in \N$ tal que los términos $x_n$ con $n\geq n_0$ satisfacen que $|x_n-\ell|<\epsilon$, $\forall \epsilon>0$.

Llamamos a $\ell$ el límite de la secuencia y escribimos $\lim X = \ell$.

%Notemos que por LE5(a) se cumple que $x_n=\ell$ con $n\geq n_0$. Esto es falso.
Si la sucesión no es convergente, entonces es divergente.

\textbf{Definición:} Una secuencia $X$ está acotada si $\exists M\in \R^+$ tal que $|x_n|\leq M, \forall n\in \N$.

\subsection*{Lista de Ejercicios 10 (LE10)}

Demuestre lo siguiente:

\begin{enumerate}[label=\alph*),font=\bfseries]
    \item Toda secuencia convergente está acotada.
\end{enumerate}

\subsubsection*{Demostración}

\begin{enumerate}[label=\alph*),font=\bfseries]
    \item Sea $X$ una secuencia convergente. Por definición, $\exists n_0 \in \N$ tal que los términos $x_n$ con $n\geq n_0$ satisfacen que $|x_n-\ell|<\epsilon, \forall \epsilon>0$. Sea $\epsilon_0>0$ arbitrario pero fijo, entonces \begin{align*}
        |x_n - \ell| &< \epsilon_0 \\
        |x_n - \ell| + |\ell| &< \epsilon_0 + |\ell|
    \end{align*}\\
    Luego, notemos que $|x_n| = |x_n-\ell+\ell|$, y por la desigualdad del triángulo, \[|x_n-\ell+\ell| \leq |x_n-\ell| + |\ell|\]\\ Por transitividad, $|x_n|< \epsilon_0 + |\ell|$, lo que implica que $\{x_n\}$ con $n\geq n_0$ está cotado superiormente.
    
    Por otra parte, notemos que el conjunto de índices $n<n_0$ está acotado superiormente, y por esto, $\{x_{n<n_0}\}$ es finito y por tanto, tiene supremo. %proof https://math.stackexchange.com/questions/548806/a-finite-set-always-has-a-maximum-and-a-minimum
    
    Finalmente, el conjunto $\{x_{n<n_0}\} \cup \{\epsilon_0 + |\ell|\}$ está acotado superiormente, por lo que $X$ está acotada. \qed
\end{enumerate}

\end{document}
