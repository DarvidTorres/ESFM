\documentclass[11pt]{article}

\usepackage{amsfonts, amssymb, amsmath, amsthm, enumitem, longdivision}

%\usepackage[top=1.4in, bottom=1in, left=1.43in, right=1.39in]{geometry}
\usepackage[top=0.6in,bottom=0.6in,right=1in,left=1in]{geometry}
\usepackage{xcolor,graphicx}
\usepackage[subtle]{savetrees}

%Conjuntos de números
\newcommand{\N}{\mathbb{N}}
\newcommand{\Z}{\mathbb{Z}}
\newcommand{\Q}{\mathbb{Q}}
\newcommand{\R}{\mathbb{R}}
\newcommand{\C}{\mathbb{C}}

%Dark mode
%\pagecolor[rgb]{0,0,0} %black
%\color[rgb]{1,1,1} %white


%the code below manipulates space around \align environment

\usepackage{etoolbox}
\newcommand{\zerodisplayskips}{%
  \setlength{\abovedisplayskip}{10pt}%
  \setlength{\belowdisplayskip}{-10pt}%
  \setlength{\abovedisplayshortskip}{10pt}%
  \setlength{\belowdisplayshortskip}{-10pt}}
\appto{\normalsize}{\zerodisplayskips}
\appto{\small}{\zerodisplayskips}
\appto{\footnotesize}{\zerodisplayskips}

%the code above manipulates space around \align environment

\setlength{\parindent}{0pt} %no indent for the document
\setlength{\parskip}{1em} %add space between paragraphs
\pagestyle{empty}

\newtheorem{Teorema}{Teorema}
\renewcommand{\qed}{\hfill$\blacksquare$}

%bold all lists$
\setlist[enumerate]{font=\bfseries}

\begin{document}

\title{Álgebra I}
\author{Darvid}
\date{\today}
\maketitle
\thispagestyle{empty}

%PROPIEDADES DE LOS NÚMEROS ENTEROS

\section{Capítulo 0. Propiedades de los números enteros}

\begin{enumerate}[label=P\arabic*.]
    \item $b|b$, para cada $b \in \Z$.
    \item $b|0$, para cada $b \in \Z$.
    \item $1|a$ y $-1|a$, para cada $a \in \Z$.
    \item $0|a \Longleftrightarrow a=0$.
    \item Si $b|1$, entonces $b=\pm 1$.
    \item Si $b|a$ y $a|b$, entonces $a=\pm b$.
    \item Si $b|a$ y $a|c$, entonces $b|c$.
    \item Si $b|a$ y $b|c$, entonces $b|a+c$ y $b|a-c$.
    \item Si $b|a$, entonces $b|ac \ \forall c \in \Z$.
    \item Si $b|a$ y $b|c$, entonces $b|as+ct \ \forall s,t \in \Z$.
    \item $b|a \Longleftrightarrow b|-a \Longleftrightarrow -b|a \Longleftrightarrow -b|-a$.
    \item $b|a \Longleftrightarrow b \Big||a| \Longleftrightarrow |b|\Big| a \Longleftrightarrow |b|\Big||a|$.
    \item Si $b|a+c$ y $b|a$, entonces $b|c$.
\end{enumerate}

%SOLUCIÓN PROPIEDADES DE LOS NÚMEROS ENTEROS | DEMOSTRACIÓN PROPIEDADES

\textbf{Demostración:}
\begin{enumerate}[label=P\arabic*.]
    \item Notemos que $b=b \cdot 1, \ \forall b\in \Z$. Entonces $b|b$, para cada $b \in \Z$.
    \item Notemos que $0=b \cdot 0, \ \forall b\in \Z$. Entonces $b|0$, para cada $b \in \Z$.
    \item Notemos que $a=1 \cdot a$ y $a = (-1) \cdot (-a), \ \forall a\in \Z$. Entonces $1|a$ y $-1|a$, para cada $a \in \Z$.
    \item 
      \begin{enumerate}[label=\roman*)]
      \item Si $0|a$, entonces $\exists q \in \Z$ tal que $a=0 \cdot q$, por lo que $a=0$.
      \item Si $a=0$, se verifica que $\exists q \in \Z$ tal que $a = 0 \cdot q$, por lo que $0|a$.
      \end{enumerate}
    \item Si $b|1$, entonces $\exists q \in \Z$ tal que $1 = bq$. Como $1>0$ se verifica que $|bq|=bq$. Además $|bq|=|b||q|$, por lo que $1 = |b||q|$. Por definición $|b| \geq 0$ y $|q| \geq 0$, pero $b \neq 0$ y $q \neq 0$, entonces $|b| > 0$ y $|q| > 0$. Luego, como $|b|,|q| \in \N$ sigue que $|b| \geq 1$ y $|q| \geq 1$, pero $|b| > 1$ contradice nuestra hipótesis. Por tanto, $b= \pm 1$.
    \item Si $b|a$ y $a|b$, entonces $\exists q_1, q_2 \in \Z$ tales que $a=bq_1$ y $b=aq_2$. De este modo, $a=a(q_1q_2)$. Suponiendo que $a \neq 0$, tenemos que $1=q_1q_2$, lo cual implica que $q_1|1$ y por (A5), $q_1= \pm 1$. Por tanto, $a= \pm b$.
    \item Si $b|a$ y $b|c$, entonces $\exists q_1,q_2 \in \Z$ tales que $a=bq_1$ y $c=aq_2$. De este modo, $c=b(q_1q_2)$, es decir, $b|c$.
    \item Si $b|a$ y $b|c$, entonces $\exists q_1,q_2 \in \Z$ tales que $a=bq_1$ y $c=bq_2$. De este modo, $a+c=bq_1+bq_2$, es decir $a+c=b(q_1+q_2)$ lo que implica que $b|a+c$. Similarmente, $a-c=bq_1-bq_2$, es decir, $a-c=b(q_1-q_2)$ lo que implica que $b|a-c$.
    \item Si $b|a$, entonces $\exists q \in \Z$ tal que $a=bq$. Sea $c \in \Z$ arbitrario pero fijo. Notemos que $ac=bqc$, lo que implica que $b|ac, \ \forall c \in \Z$.
    \item Si $b|a$ y $b|c$, entonces $\exists q_1,q_2 \in \Z$ tales que $a=bq_1$ y $c=bq_2$. Sean $s,t\in \Z$ arbitrarios pero fijos, entonces $as= bq_1s$ y $ct=bq_2t$. De este modo, $as+ct=bq_1s+bq_2t$, es decir $as+ct=b(q_1s+q_2t)$, lo cual implica que $b|as+ct$.
    \item
      \begin{enumerate}[label=\roman*), font=\bfseries]
      \item Si $b|a$, entonces $\exists q \in \Z$ tal que $a=bq$. Notemos que   $-a=-bq$, es decir $-a=b(-q)$ lo que implica que $b|-a$.
      \item Si $b|-a$, entonces $\exists q \in \Z$ tal que $-a=bq$. Notemos   que $a=-bq$, lo que implica que $-b|a$.
      \item Si $-b|a$, entonces $\exists q \in \Z$ tal que $a=-bq$. Notemos   que $-a=-(-bq)$, es decir, $-a=-b(-q)$, lo que implica que $-b|-a$.
      \item Si $-b|-a$, entonces $\exists q \in \Z$ tal que $-a=-bq$. Notemos que $a=bq$, lo que implica que $b|a$.
      \end{enumerate}
    \item
      \begin{enumerate}[label=\roman*)] 
      \item Si $b|a$, entonces $\exists q\in \Z$ tal que $a=bq$.
        \begin{enumerate}[label=\alph*),font=\bfseries]
          \item Si $a \geq 0$, entonces $|a|=a$, por lo que $a=bq$, lo cual implica que $b \Big||a|$.
          \item Si $a \leq 0$, entonces $|a|=-a$, por lo que $-a=b(-q)$, lo cual implica que $b \Big||a|$.
        \end{enumerate}
      \item Si $b \Big||a|$, entonces $\exists q \in \Z$ tal que $|a|=bq$.
        \begin{enumerate}[label=\alph*),font=\bfseries]
          \item Si $a \geq 0$ y $b \geq 0$, entonces $|a|=a$ y $|b|=b$, por lo que $a=|b|q$, lo cual implica que $|b| \Big| a$.
          \item Si $a \geq 0$ y $b < 0$, entonces $|a|=a$ y $|b|=-b$, por lo que $-|a|=-bq$, es decir, $-a=|b|q$. Notemos que $a=|b|(-q)$, lo cual implica que $|b| \Big| a$.
          \item Si $a < 0$ y $b \geq 0$, entonces $|a|=-a$ y $|b|=b$, por lo que $-|a|=b(-q)$, es decir, $a=|b|(-q)$, lo cual implica que $|b| \Big| a$.
          \item Si $a<0$ y $b<0$, entonces $|a|=-a$ y $|b|=-b$, por lo que $-|a|=-bq$, es decir, $a=|b|q$, lo cual implica que $|b| \Big| a$.
        \end{enumerate}
      \item Si $|b| \Big| a$, entonces $\exists q \in \Z$ tal que $a= |b|q$
        \begin{enumerate}[label=\alph*),font=\bfseries]
          \item Si $a \geq 0$, entonces $|a|=a$, por lo que $|a|=|b|q$, lo cual implica que $|b| \Big| |a|$.
          \item Si $a <0$, entonces $|a|=-a$, por lo que $-a=-|b|q$, es decir, $|a|=|b|(-q)$, lo cual implica que $|b| \Big| |a|$.
        \end{enumerate}
      \item Si $|b| \Big| |a|$, entonces $\exists q \in \Z$ tal que $|a|=|b|q$.
      \begin{enumerate}[label=\alph*),font=\bfseries]
        \item Si $a \geq 0$ y $b \geq 0$, entonces $|a|=a$ y $|b|=b$, por lo que $a=bq$, lo cual implica que $b|a$.
        \item Si $a \geq 0$ y $b < 0$, entonces $|a|=a$ y $|b|=-b$, por lo que $a=-bq$, es decir, $a=b(-q)$, lo cual implica que $b|a$.
        \item Si $a < 0$ y $b \geq 0$, entonces $|a|=-a$ y $|b|=b$, por lo que $-a=bq$, es decir $a=b(-q)$, lo cual implica que $b|a$.
        \item Si $a<0$ y $b<0$, entonces $|a|=-a$ y $|b|=-b$, por lo que $-a=-bq$, es decir, $a=bq$, lo cual implica que $b|a$.
      \end{enumerate}
      \end{enumerate}
      \item Si $b|a+c$ y $b|a$, entonces $\exists q_1,q_2 \in \Z$ tales que $a+c=bq_1$ y $a=bq_2$. Luego, $(bq_2)+c=bq_1$, es decir, $c=b(q_1-q_2)$, lo que implica que $b|c$.
    \end{enumerate}


\textbf{Definición}. Sean $a,b,c\in \Z$. Decimos que $c$ es combinación lineal de $a$ y $b$ si existen $x,y\in \Z$ tales que $c=ax+by$.

%EJERCICIOS CAPÍTULO 0

\section{Ejercicios}

\begin{enumerate}[label=1.\arabic*]
  \item  Pruebe que $29$ es combinación lineal de $5$ y $7$.
  \item Escriba a $50$ en dos formas diferentes como combinación lineal de $5$ y $2$.
  \item Si $d|a$, $d|b$ y $d \nmid c$, pruebe que $c$ no es combinación lineal de $a$ y $b$.
  \item Pruebe que $64$ no es combinación lineal de $10$ y $25$.
  \item Encuentre un entero $m$ que no sea combinación lineal de $28$ y $49$.
  \item Si $m$ divide a cualquier combinación lineal de $a$ y $b$, pruebe que $m|a$ y $m|b$.
  \item Decida si la ecuación $153=34x+51y$ tiene soluciones enteras $x$ y $y$.
  \item Si $c$ es impar, pruebe que la ecuación $c=14x+72y$ no tiene soluciones enteras $x$ y $y$.
\end{enumerate}

\begin{enumerate}[start=2]
  \item Si $b|m$ para todo $m\in \Z$, pruebe que $b= \pm 1$.
  \item Si $b|a_1, b|a_2, \dots, b|a_n$, pruebe que $b|a_1+a_2+\dots +a_n$.
  \item Pruebe que
    \begin{enumerate}[label=4.\arabic*]
      \item $8|(2n-1)^2-1$, para cada $n\in\N$.
      \item $6|n^3-n$, para cada $n\in \N$.
      \item $9|n^3+(n+1)^3+(n+2)^3$, para cada $n \in \N$.
      \item $133|11^{n+2}+12^{2n+1}$, para cada $n\in \N$.
      \item Si $a,b,c$ son dígitos, entonces $143$ divide al número   (cifrado) $abcabc$.
    \end{enumerate}
  \item Si $a,b\in \Z$, pruebe que $a-b|a^n-b^n$, para cada $n\in \N$.
  \item Sean $a,b\in \Z$, con $b \neq 0$. Pruebe que $b|a$, si y solo si, el residuo de dividir $a$ por $b$, es $r=0$.
  \item Aplicando el algoritmo de división , encuentre $q$ y $r$ para escribir $a=bq+r$ en los siguientes casos:
    \begin{enumerate}[label=7.10]
      \item $a=m^3+3m^2+3m+2$ y $b=m+1 \ (m>0)$.
    \end{enumerate}
  \item Pruebe que $(a,b)=(|a|,|b|)$.
  \item Aplicando el algoritmo de Euclides y el ejercicio anterior, encuentre el mcd de:
    \begin{enumerate}[label=9.5]
      \item $a=764$ y $b=-866$.
    \end{enumerate}
  \item Si $(a,b)=1$, pruebe que la ecuación $c=ax+by$ tiene soluciones enteras $x$ y $y$, para cada $c\in\Z$.
  \item Sean $a,b,c \in\Z$. Si $d=(a,b)$, pruebe que la ecuación $c=ax+by$ tiene soluciones enteras, si y solo si, $d|c$.
  \item Si $d>0$ es tal que $d|a, \ d|b$ y $d=as+bt$, pruebe que $d=(a,b)$.
  \item Si $d=(a,b)$ y $d=as+bt$, pruebe que $(s,t)=1$. [¿Son únicos $s$ y $t$?].
  \item Si $d=(a,b)$, $a=bq_1$ y $b=dq_2$, pruebe que $(q_1,q_2)=1$.
  \item Si $c|a$ y $(a,b)=1$, pruebe que $(b,c)=1$.
  \item Si $a|c, \ b|c$ y $d=(a,b)$, pruebe que $ab|cd$.
  \item Si $(a,b)=1$ y $c \neq 0$, pruebe que $(a,bc)=(a,c)$.
  \item Si $k>0$, pruebe que $(ak,bk)=k(a,b)$.
  \item Si $k \neq 0$, pruebe que $(ak,bk)=|k|(a,b)$.
  \item Si $(a,b)=1$, pruebe que $(a+b,a-b)=1$ ó $2$.
  \item Si $(a,b)=1$, pruebe que $(a^m,b^n)=1$ para todo $n,m\in \N$.
  \item Si $(a,b)=k$, pruebe que $(a^n,b^n)=k^n$ para todo $n\in\N$.
  \item Sean $m,n,k\in \N$. Si $mn=k^2$ y $(m,n)=1$, pruebe que $m=a^2$ y $n=b^2$ para algunos $a,b\in \N$.
  \item Si $(a,c)=1$ y $(b,c)=1$, pruebe que $(ab,c)=1$.
  \item Si $b^2|a^2$, prube que $b|a$.
  \item Si $b^n|a^n$, pruebe que $b|a$.
  \item Si $a\in \N$ y $a\neq k^2$ para todo $k\in \N$, pruebe que $\sqrt{a} \notin \Q$.
  \item Si $a\in \N$ y $a\neq k^n$ para todo $k\in \N$, pruebe que $\sqrt[n]{a} \notin \Q$.
  \item Si $a_1,a_2,\dots, a_n$ son dígitos, pruebe que $9|a_1a_2\dots a_n$, si y solo si, $9|a_1+a_2+\dots a_n$ ($a_1a_2\dots a_n$ es un número cifrado). Sugerencia: Pruebe y use que $9|10^n-1$, para cada $n\in\N$.
\end{enumerate}

\begin{enumerate}[label=30.1]
  \item Si $d=(a_0,a_1,\dots,a_n)$, pruebe que $d$  es único.
\end{enumerate}

\begin{enumerate}[start=31]
  \item Sean $a,b,c\in\Z$, no todos cero. Pruebe que $(a,b,c)=\big( (a,b),c \big)$.
  \item Sean $a_0,a_1,\dots,a_n \in \Z$, no todos cero. Pruebe que $(a_1,a_2,\dots,a_n)=\big(|a_1|,|a_2|,\dots,|a_n|\big)$.
  \item Sean $a,b \in \Z$ con $a\neq 0$ y $b \neq 0$. Decimos que $m\in\Z$, $m>0$ es mínimo común múltiplo (mcm) de $a$ y $b$, y escribimos $m=[a,b]$ ó $m=$ mcm $\{a,b\}$, si:
  \begin{enumerate}[label=\roman*)]
    \item $a|m$ y $b|m$.
    \item Si $a|s$ y $b|s$ para algún $s\in\Z$, entonces $m|s$.
  \end{enumerate}
    \begin{enumerate}[label=33.1]
      \item Si $m=[a,b]$, pruebe que $m$ es único. 
    \end{enumerate}
    \begin{enumerate}[label=33.2]
      \item Dados $a,b\in \Z-\{0\}$, pruebe que existe $m=[a,b]$.
    \end{enumerate}
    \begin{enumerate}[label=33.3]
      \item Pruebe que $[a,b]=\big[|a|,|b| \big]$.
    \end{enumerate}
    \begin{enumerate}[label=33.4]
      \item Si $a>0$ y $b>0$ pruebe que $[a,b]=\frac{a \cdot b}{(a,b)}$.
    \end{enumerate}
    \begin{enumerate}[label=33.5]
      \item Si $k>0$, pruebe que $[ak,bk]=k[a,b]$.
    \end{enumerate}
\end{enumerate}

\begin{enumerate}[start=37]
  \item Si $p$ es primo y $p|a_1 \cdot a_2 \cdot \dots \cdot a_n$, pruebe que $p|a_i$ para algún $i=1,2, \dots, n$.
  \item Si $a\in \Z$ y $a<-1$, pruebe que existen primos $p_1,p_2,\dots,p_n$ tales que $a=-p_1\cdot p_2 \cdot \dots \cdot p_n$.
\end{enumerate}

\begin{enumerate}[start=40]
  \item Sea $n\in \N$. Si $2^n-1$ es primo, pruebe que $n$ es primo.
\end{enumerate}

\begin{enumerate}[start=43]
  \item Si $p$ es un número primo y $n\in \N$, pruebe que la suma de los divisores positivos de $p^{n-1}$ es $\frac{p^n-1}{p-1}$.
  \item Pruebe que el conjunto de números primos no es finito.
\end{enumerate}


%SOLUCIÓN EJERCICIOS CAPÍTULO 0 | DEMOSTRACIONES CAPÍTULO 0

\textbf{Demostración:}
\begin{enumerate}[label=1.\arabic*]
  \item Notemos que $29=(5)(3)+(7)(2)$, es decir, existen $s,t \in \Z$ tales que $29=5s+7t$, por lo que $29$ es una combinación lineal de $5$ y $7$.
  \item $50=(5)(10)+(2)(0)$ y $50=(5)(2)+(2)(20)$.
  \item Supongamos que $c$ es combinación lineal de $a$ y $b$, entonces $\exists s,t\in \Z$ tales que $c=as+bt$. Si $d|a$ y $d|b$, entonces $\exists q_1,q_2\in \Z$ tales que $a=dq_1$ y $b=dq_2$. Luego, $as=dq_1s$ y $bt=dq_2t$. De este modo, $as+bt=dq_1s+dq_2t$, es decir, $as+bt=d(q_1s+q_2t)$, lo cual implica que $d|c$, pero esto contradice nuestra suposición. Por tanto, $c$ es combinación lineal de $a$ y $b$.
  \item Supongamos que $64$ es combinación lineal de $10$ y $25$, entonces $\exists s,t\in \Z$ tales que $64=10s+25t$. Notemos que $64=5(2s+5t)$, lo cual implica que $5|64$, pero $64$ no satisface la divisibilidad por $5$. Por tanto, $64$ no es combinación lineal de $10$ y $25$.
  \item $m=1$.
  \item Si $m|as+bt \ \forall s,t\in\Z$, entonces elegimos $s=1$ y $t=0$, por lo que $m|a$. Luego, elegimos $s=0$ y $t=1$, por lo que $m|b$.
  \item Dada la ecuación $153=34x+51y$, notemos que $153=(17)(9)$ y $34x+51y=17(2x+3y)$, entonces $9=2x+3y$, por lo que la ecuación tiene soluciones enteras $x=3$ y $y=1$.
  \item Dada la ecuación $c=14x+72y$, tenemos que $c=2(7x+36y)$, lo cual implica que $2|c$, pero esto contradice la hipótesis de que $c$ es impar.
\end{enumerate}

\begin{enumerate}[start=2]
  \item Si $b|m \ \forall m\in\Z$, elegimos $m=1$, por lo que $b|1$ y, por \textbf{(P5)}, sigue que $b= \pm 1$.
  \item Procedamos por inducción sobre el número de elementos.
    \begin{enumerate}[label=\roman*)]
      \item Si $b|a_1$ y $b|a_2$, por \textbf{(P8)} se verifica que $b|a_1+a_2$.
      \item Supongamos que si $b|a_1, b|a_2, \dots , b|a_k$, entonces $b|a_1+a_2+\ldots+a_k$.
      \item Si $b|a_1,b|a_2,\dots,b|a_k, b|a_{k+1}$, por hipótesis de inducción tenemos que $b|a_1+a_2+\ldots+a_k$, y dado que $b|a_{k+1}$ por \textbf{(P8)} se verifica que $b|a_1+a_2+\ldots+a_k+a_{k+1}$.
    \end{enumerate}
    Por tanto, si $b|a_1, b|a_2, \dots, b|a_n$, entonces $b|a_1+a_2+\dots +a_n, \ \forall n \in \N$ con $n \geq 2$.
\end{enumerate}

\begin{enumerate}[label=4.\arabic*]
  \item Procederemos por inducción en $n$.
    \begin{enumerate}[label=\roman*)]
      \item Verificamos que se cumple para $n=1$.\\ En efecto, $8|(2\cdot 1-1)^2-1$, es decir, $8|0$, lo cual se verifica por \textbf{(P2)}.
      \item Supongamos que se cumple para $n=k$, es decir, supongamos que $8|(2k-1)^2-1$ \\ Lo que implica que $\exists q\in \Z$ tal que
      \begin{align*}
        8q &= (2k-1)^2-1 \\
        &= (4k^2-4k+1)-1  \\
        &= 4k^2-4k
      \end{align*}
      \item Luego, si $n=k+1$, tenemos que
      \begin{align*}
        4(k+1)^2-4(k+1) &= 4(k^2+2k+1)-4k-4\\
        &= 4k^2+8k+4-4k-4\\
        &= 4k^2-4k+8k\\
        &= 8q+8k && \text{Por hipótesis de inducción}\\
        &= 8(q+k) \\
      \end{align*}
      Finalmente, $8|8(q+k)$ se verifica ya que $q+k\in \Z$.
      Por tanto, $8|(2n-1)^2-1$, para cada $n\in\N$.
    \end{enumerate}
  \item Procederemos por inducción en $n$.
    \begin{enumerate}[label=\roman*)]
      \item Verificamos que se cumple para $n=1$.\\ En efecto, $6|(1)^3-1$, es decir, $6|0$, lo cual se verifica por \textbf{(P2)}.
      \item Supongamos que se cumple para $n=k$, es decir, supongamos que $6|k^3-k$.
      \item Luego, si $n=k+1$, tenemos que
      \begin{align*}
        (k+1)^3-(k+1) &= k^3+3k^2+3k+1-k-1 \\
        &= k^3-k+3k^2+3k
      \end{align*}
      \end{enumerate}
    Notemos que por hipótesis de inducción $6|k^3-k$, y si logramos demostrar que $6|3k^2+3k$, por \textbf{(P8)} garantizaríamos que $6|k^3-k+3k^2+3k$. Así, demostraremos que $6|3k^2+3k \ \forall n\in \N$ por inducción en $n$.
    \begin{enumerate}[label=\roman*)]
      \item Verificamos que se cumple para $n=1$. \\ Si $n=1$, tenemos que $6|3(1)^2+3(1)$, es decir, $6|6$, lo que se valida por \textbf{(P1)}.
      \item Supongamos que se cumple para $n=k$, es decir, supongamos que $6|3k^2+3k$. Lo que implica que $\exists q\in \Z$ tal que $6q=3k^2+3k$.
      \item Luego, si $n=k+1$, tenemos que
      \begin{align*}
        3(k+1)^2+3(k+1) &= 3(k^2+2k+3)+3k+3  \\
        &= 3k^2+6k+9+3k+3 \\
        &= 6q+6k+12 \\
        &= 6(q+k+2) \\
      \end{align*}
      Es decir $6|6(q+k+2)$ lo cual es verdadero ya que $q+k+2 \in \Z$. Por tanto, $6|3k^2+3k$, lo que a su vez por \textbf{(P8)}, implica que $6|n^3-n$, para cada $n\in \N$.
    \end{enumerate}
  \item Procederemos por inducción en $n$.
    \begin{enumerate}[label=\roman*)]
      \item Verificamos que se cumple para $n=1$. \\ Es claro que $9|(1)^3+(1+1)^3+(1+2)^3$, es decir, $9|1+8+27$, osea $9|36$, lo cual es verdadero.
      \item Supongamos que se cumple para $n=k$, es decir, supongamos que \[9|k^3+(k+1)^3+(k+2)^3\] \\ Lo que implica que $\exists q \in \Z$ tal que $9q=k^3+(k+1)^3+(k+2)^3$.
      \item Luego, si $n=k+1$, tenemos que
      \begin{align*}
        (k+1)^3+\big((k+1)+1\big)^3+\big((k+1)+2\big)^3 &= (k+1)^3+(k+2)^3+(k+3)^3 \\
        &= (k+1)^3+(k+2)^3+k^3+9k^2+27k+27 \\
        &= 9q+9k^2+27k+27 && \text{Por (ii)} \\
        &= 9(q+k^2+3k+3)
      \end{align*}

      Finalmente, $9|9(q+k^2+3k+3)$ se verifica ya que $(q+k^2+3k+3)\in \Z$. Por tanto, $9|n^3+(n+1)^3+(n+2)^3$, para cada $n \in \N$.
    \end{enumerate}
    \item Procederemos por inducción en $n$.
      \begin{enumerate}[label=\roman*)]
        \item Verificamos que se cumple para $n=1$. \\ Es claro que
        \begin{align*}
          133 &|11^{1+2}+12^{2(1)+1} \\
          133 &|11^{3}+12^{3} \\
          133 &|3059 \\
          133 &| 133(23)
        \end{align*}
      \item Supongamos que se cumple para $n=k$, es decir, supongamos que $133|11^{k+2}+12^{2k+1}$. \\ Lo que implica que $\exists q \in \Z$ tal que $133q=11^{k+2}+12^{2k+1}$.
      \item Luego, si $n=k+1$, tenemos que
        \begin{align*}
          11^{(k+1)+2}+12^{2(k+1)+1} &= 11^{k+3}+12^{2k+3} \\
          &= 11^{k+2} \cdot 11 + 12^{2k+1} \cdot 12^2 \\
          &= 11^{k+2} (144-133) + 12^{2k+1} \cdot 144 \\
          &= 144 \cdot 11^{k+2} - 133 \cdot 11^{k+2} + 144 \cdot 12^{2k+1} \\
          &= 144 \big( 11^{k+2} + 12^{2k+1} \big) - 133 \cdot 11^{k+2} \\
          &= 144 (133q) - 133 \cdot 11^{k+2} && \text{Por hipótesis de inducción}\\
          &= 133 (144q-11^{k+2})
        \end{align*}

        Finalmente, $133|133(144q-11^{k+2})$ se verifica ya que $(144q-11^{k+2})\in \Z$. Por tanto, $133|11^{n+2}+12^{2n+1}$, para cada $n\in \N$.
        \end{enumerate}
      \item Notemos que
      \begin{align*}
      abcabc &= 100000a+10000b+1000c+100a+10b+c \\
        &= 100100a+10010b+1001c\\
        &= (1001 \cdot 100)a + (1001 \cdot 10)b + 1001c \\
        &= 1001(100a+10b+c) \\
        &= 143(7)(100a+10b+c)\\
        &= 143(700a+70b+7c)\\
      \end{align*}
      Luego, tenemos que $143|143(700a+70b+7c)$, lo cual es verdadero. Por tanto, $143|abcabc$.
\end{enumerate}

\begin{enumerate}[start=5]
  \item Notemos que \[ a^n-b^n = (a-b)(a^{n-1}+ a^{n-2}b+ \ldots + ab^{n-2} + b^{n-1}) \] \\
  Lo cual implica que $a-b|a^n-b^n$.
  \item Por el \textbf{Teorema 0.2.1} sabemos que existen $q,r \in \Z$ únicos tales que $a=bq+r$ con $0 \leq r < |b|$. Luego,
    \begin{enumerate}[label=\roman*)]
      \item Si $b|a$, entonces $\exists k \in \Z$ tal que $a=bk$. Notemos que $a=bk+0$. Como $a=bq+r$ y $q$ y $r$ son únicos, sigue que $r=0$.
      \item Si $a=bq+r$ y $r=0$, entonces $a=bq$, lo cual implica que $b|a$.
    \end{enumerate}
\end{enumerate}

\begin{enumerate}[label=7.10]
  \item $q=m^2+2m+1$ y $r=1$.
\end{enumerate}

\begin{enumerate}[start=8]
  \item
  \begin{enumerate}[label=\roman*)]
    \item Si $ a \geq 0$ y $b \geq 0$, entonces $|a|=a$ y $|b|=b$. Por lo que $(a,b)=(|a|,|b|)$.
    \item Si $a \geq 0$ y  $b<0$, entonces $|a|=a$ y $|b|=-b$. Sea $d=(a,b)$, entonces $d|a$ y $d|b$ por lo que $\exists q_1, q_2\in \Z$ tales que $a=dq_1$ y $b=dq_2$, es decir $|a|=dq_1$ y $-b=-dq_2$, osea $|b|=d(-q_2)$, lo que implica que $d \Big| |a|$ y $d \Big| |b|$. Sigue que $(a,b)=(|a|,|b|)$.
    \item Si $a<0$ y $b<0$, entonces $|a|=-a$ y $|b|=-b$.  Sea $d=(a,b)$, entonces $d|a$ y $d|b$ por lo que $\exists q_1, q_2\in \Z$ tales que $a=dq_1$ y $b=dq_2$. Luego, $-a=-dq_1$ y $-b=-dq_2$, es decir, $|a|=d(-q_1)$ y $|b|=d(-q_2)$, lo que implica que $d \Big| |a|$ y $d \Big| |b|$. Por tanto, $(a,b)=(|a|,|b|)$.
  \end{enumerate} 
\end{enumerate}

\begin{enumerate}[label=9.5]
  \item Por el ejercicio $8$, $(764,-866)=(764,866)$. Aplicando el algoritmo de Euclides, tenemos:
  \vspace{0.5 cm}

  \intlongdivision{866}{764}
  \quad
  \intlongdivision{764}{102}
  \quad
  \intlongdivision{102}{50}
  \quad
  \intlongdivision{50}{2} \\
  Por tanto, $2=(764,-866)$.
\end{enumerate}

\begin{enumerate}[start=10]
  \item Si $(a,b)=1$, entonces $\exists s,t\in\Z$ tales que $1=as+bt$. Sea $c \in\Z$ arbitrario pero fijo, entonces $c=asc+btc$, por lo que la ecuación $c=ax+by$ tiene soluciones enteras $x=sc$ y $tc$.
  \item 
    \begin{enumerate}[label=\roman*)]
      \item Si $d=(a,b)$, entonces $d|a$ y $d|b$.   Supongamos que $d \nmid c$, por el ejercicio 1.3,   sigue que $\nexists x,y\in \Z$ tales que $c=ax+by$,   entonces, por contraposición, $d|c$.
      \item Si $d=(a,b)$, entonces  $\exists s,t\in \Z$ tales que $d=as+bt$. Si $d|c$,   entonces $\exists q\in \Z$ tal que $c=dq$. Luego,   $c=(as+bt)q$, osea $c=asq+btq$, por lo que la   ecuación $c=ax+by$ tiene soluciones enteras $x=sq$  y $y=tq$.
    \end{enumerate}
  \item Tenemos que $d|a$ y $d|b$ con $d>0$. Sea $c\in \Z$ tal que $c|a$ y $c|b$. Por \textbf{(P9)}, $c|as$ y $c|bt$, y por \textbf{(P8)}, $c|as+bt$. Por hipótesis $d=as+bt$, entonces $c|d$. Por tanto, $d=(a,b)$.
  \item Si $d=(a,b)$ y $d=as+bt$, entonces $d|a$ y $d|b$, es decir $\exists q_1, q_2 \in\Z$ tales que $a=dq_1$ y $b=dq_2$. Sigue que $d=(dq_1)s+(dq_2)t$, osea $d=d(sq_1+tq_2)$. Como $d>0$, tenemos que $1=sq_1+tq_2$. Es claro que $1$ es combinación lineal de $s$ y $t$, además es el mínimo entero positivo que satisface esta combinación lineal, entonces $(s,t)=1$. Finalmente, la solución de \textbf{Ej. 1.2} es un contraejemplo de que $s$ y $t$ no son únicos.
  \item Si $d=(a,b)$, entonces $\exists s,t\in\Z$ tales que $d=as+bt$. Además, por hipótesis, $a=dq_1$ y $b=dq_2$, lo que implica que $d=(dq_1)s+(dq_2)t$, osea $d=d(q_1s+q_2t)$. Como $d>0$, sigue que $1=q_1s+q_2t$. Es claro que $1$ es combinación lineal de $q_1$ y $q_2$, y es el mínimo entero positivo que satisface esta propiedad, entonces, $(q_1,q_2)=1$.
  \item Si $c|a$, entonces $\exists q\in\Z$ tal que $a=cq$. Además, por hipótesis $(a,b)=1$, lo que implica que $1=as+bt$. Notemos que $1=(cq)s+bt$, osea $1=c(qs)+bt$. Es claro que $1$ es combinación lineal de $c$ y $b$, además es el mínimo entero positivo que satisface esta combinación lineal, entonces,  por el Algoritmo de Euiclides, $(b,c)=1$.
  \item Si $a|c$ y $b|c$, entonces $\exists q_1, q_2 \in\Z$ tales que $c=aq_1$ y $c=bq_2$. Además, por hipótesis $d=(a,b)$, entonces $\exists s,t\in \Z$ tales que $d=as+bt$. Luego $cd=c(as+bt)$, osea $cd=asc+btc$. Notemos que $cd=as(bq_2)+bt(aq_1)$, por lo que $cd=ab(sq_2)+ab(tq_1)$. Finalmente, $cd=ab(sq_2+tq_1)$, lo cual implica que $ab|cd$.
  \item Si $(a,b)=1$ entonces $1|a$ y $1|b$. Luego, por \textbf{(P9)}, $1|bc$. Además, si $\exists d \in \Z$ tal que $d|a$ y $d|b$, entonces tendríamos que $d|1$, y por \textbf{(P5)},  $d= \pm 1$. De este modo, $1=(a,bc)$, es decir, $(a,b)=(a,bc)$.
  \item Tenemos que $\exists d=(a,b)$ y $d$ es el mínimo entero positivo para el cual $\exists s,t\in\Z$ tales que $d=as+bt$. Notemos que $kd=k(as+bt)$, osea $kd=(ak)s+(bk)t$. Como $k>0$, se verifica que $kd$ es el mínimo entero postivo que satisface una combinación lineal de $ak$ y $bk$, lo que implica que $(ak,bk)=kd$, osea $(ak,bk)=k(a,b)$.
  \item Tenemos que $\exists d=(a,b)$ y $d$ es el mínimo entero positivo para el cual $\exists s,t\in\Z$ tales que $d=as+bt$. Por definición, $|k|\geq 0$, pero $k \neq 0$, así, se sigue que $|k|>0$. Notemos que $|k|d=|k|(as+bt)$, osea $|k|d=a|k|s+b|k|t$. Observamos que:
    \begin{enumerate}[label=\roman*)]
      \item Si $k<0$, tenemos que $|k|=-k$, por lo que  $|k|d=a(-k)s+b(-k)t$, es decir, $|k|d=ak(-s)+bk  (-t)$.
      \item Si $k>0$, entonces $|k|=k$, por lo que $|k| d=(ak)s+(bk)t$.
    \end{enumerate}
  En cualquier caso, se verifica que $|k|d$ es el mínimo entero postivo que satisface una combinación lineal de $ak$ y $bk$, lo que implica que $(ak, \ bk)=|k|d$, osea $(ak,bk)=|k|(a,b)$.
  \item Sea $k=(a+b,a-b)$, entonces $k|a+b$ y $k|a-b$, por lo que $\exists q_1,q_2 \in \Z$ tales que $(a+b)=kq_1$ y $(a-b)=kq_2$. Notemos que
    \begin{align*}
      (a+b)+(a-b) &= kq_1+kq_2 \\
      2a &= k(q_1+q_2) \\
    \end{align*}
  Similarmente, 
    \begin{align*}
      (a+b)-(a-b) &= kq_1-kq_2 \\
      2b &= k(q_1-q_2) \\
    \end{align*}
  De este modo, $k|2a$ y $k|2b$, y por \textbf{(P9)} se verifica que $k|2as$ y $k|2bt$. Además, por \textbf{(P8)} se cumple que $k|2as+2bt$. Luego, por hipótesis, $1=(a,b)$, lo que implica que $\exists s,t\in \Z$ tales que $1=as+bt$, así $2=2as+2bt$. Finalmente, $k|2$, osea $(a+b,a-b)=1$ o $2$.
  \item Tenemos que $(a,b)=1$, entonces, por la proposición $0.42$, $(a,b^n)=1 \ \forall n\in \N$. Luego, $(a,b^n)=(b^n,a)$, osea $(b^n,a)=1$. De este modo, por la proposición $0.42$, $(b^n,a^m)=1 \ \forall m\in \N$. Finalmente, $(b^n,a^m)=(a^m,b^n)$, es decir, $(a^m,b^n)=1 \ \forall m,n\in \N$.
  \item Tenemos que $(a,b)=k$, entonces $k|a$ y $k|b$, por esto $\exists q_1, q_2 \in \Z$ tales que $a=kq_1$ y $b=kq_2$. Luego, por \textbf{Ej. 14} se verifica que $(q_1,q_2)=1$. Notemos que $a^n=(kq_1)^n$ y $b^n=(kq_2)^n$, osea $a^n=k^nq_1^n$ y $b^n=k^nq_2^n$. De este modo $(a^n,b^n)=(k^nq_1^n, \ kq_2^n)$. Luego, por \textbf{Ej. 18}, se verifica que $(k^nq_1^n, kq_2^n)=k^n(q_1^n, q_2^n)$, es decir $(a^n,b^n)=k^n(q_1^n, q_2^n)$. Como $(q_1,q_2)=1$, por \textbf{Ej. 21} tenemos que $(q_1^n,q_2^n)=1$. Finalmente, observamos que $(a^n,b^n)=k^n$.
  \item Como $m$ y $n$ son primos relativos, entonces $k^2$ no es primo, lo cual implica que $\exists p_1,p_2, \dots ,p_n$ números primos tales que $k^2=p_1^2 \cdot p_2^2 \cdot \cdots p_n^2$. Además $k^2=mn$, entonces $m=p_1^2 \cdot p_2^2 \cdot \cdots p_m^2$ y $n=p_1^2 \cdot p_2^2 \cdot \cdots p_l^2$. Por tanto, $m=a^2$ y $n=b^2$, para $a=p_1 \cdot p_2 \cdot \cdots p_m$ y $b=p_1 \cdot p_2 \cdot \cdots p_l$.
  \item Tenemos que $(a,c)=1$ y $(b,c)=1$, entonces $\exists s,t,x,y\in\Z$ tales que $1=as+ct$ y $1=bx+cy$. Luego, $as=1-ct$ y $bx=1-cy$. Notemos que
    \begin{align*}
      (as)(bx) &= (1-ct)(1-cy) \\
      ab(sx) &= 1-cy-ct+ctcy \\
      ab(sx) &= 1-c(y+t-tcy) \\
      ab(sx) + c(y+t-tcy) &= 1 \\
    \end{align*}
  Es claro que $1$ es combinación lineal de $ab$ y $c$, y es el mínimo entero positivo que satisface esta propiedad. Por tanto, $(ab,c)=1$.
  \item Si $b^2|a^2$, entonces $\exists k\in\Z$ tal que $a^2=b^2k$. Luego, por el teorema fundamental de la aritmética, $\exists p_1,p_2, \dots, p_i, q_1,q_2, \dots, q_j$ primos tales que $a=p_1p_2 \cdots p_i$ y $b=q_1q_2 \cdots q_j$. Entonces, $a^2=(p_1p_2 \cdots p_i)^2$ y $b^2=(q_1q_2 \cdots q_j)^2$, osea $a^2=p_1^2p_2^2 \cdots p_i^2$ y $b^2=q_1^2q_2^2 \cdots q_j^2$. Sigue que $p_1^2p_2^2 \cdots p_i^2=kq_1^2q_2^2 \cdots q_j^2$, lo que implica que $k$ debe tener una factorización prima con potencias pares, en otras palabras, es un cuadrado perfecto. De este modo, $\sqrt{k}\in \Z$. Finalmente, de $a^2=b^2k$ obtenemos $a=b\sqrt{k}$, es decir, $b|a$.
  \item Si $b^n|a^n$, entonces $\exists k\in\Z$ tal que $a^n=b^nk$. Luego, por el teorema fundamental de la aritmética podemos escribir a $a$ y $b$ como el producto de números primos, es decir, $a=p_1^{\alpha_1} p_2^{\alpha_2} \cdots p_r^{\alpha_r}$ y $b=p_1^{\beta_1} p_2^{\beta_2} \cdots p_r^{\beta_r}$, donde $\alpha_i, \beta_i \geq 0$. Notemos que los exponentes pueden ser $0$ si algún primo ocurre en la factorización de uno de los enteros $a$ y $b$ pero no en el otro. De este modo, $a^n=p_1^{n \alpha_1} p_2^{n \alpha_2} \cdots p_r^{n \alpha_r}$ y $b^n=p_1^{n \beta_1} p_2^{n \beta_2} \cdots p_r^{n \beta_r}$. De $a^n=b^nk$ sigue que $p_1^{n \alpha_1} p_2^{n \alpha_2} \cdots p_r^{n \alpha_r}=p_1^{n \beta_1} p_2^{n \beta_2} \cdots p_r^{n \beta_r}k$, como la factorización es única, tenemos que $n\beta_i \leq n\alpha_i$ para cada $i$, lo que implica que $\beta_i \leq \alpha_i$ para cada $i$. Por tanto, $b|a$.
  \item Supongamos que $\sqrt{a}\in \Q$, entonces $\sqrt{a}=\frac{p}{q}$, con $p,q\in\Z$ y $q\neq 0$. Luego, $a=(\frac{p}{b})^2$, pero esto contradice nuestra hipótesis. Por tanto, $\sqrt{a}\notin \Q$.
  \item Supongamos que $\sqrt[n]{a}\in \Q$, entonces $\sqrt[n]{a}=\frac{p}{q}$, con $p,q\in\Z$ y $q\neq 0$. Luego, $a=(\frac{p}{q})^n$, pero esto contradice nuestra hipótesis. Por tanto, $\sqrt[n]{a}\notin \Q$.
  \item Primero demostraremos que $9|10^n-1 \ \forall n\in \N$. Procederemos por inducción sobre $n$.
    \begin{enumerate}[label=\roman*)]
      \item Verificamos que se cumple para $n=1$. \\ Es claro que $9|10^1-1$, osea  $9|9$.
      \item Supongamos que se cumple para $n=k$, es decir, supongamos que $9|10^k-1$. Esto implica que $\exists q \in \Z$ tal que $10^k-1=9q$.
      \item Luego, si $n=k+1$ tenemos que
        \begin{align*}
          10^{k+1}-1 &= (10)10^k-1\\
          &= (9+1)10^k-1\\
          &= (9)10^k+10^k-1\\
          &= (9)10^k+9q && \text{Por hipótesis de inducción}\\
          &=9(10^k+q)\\
        \end{align*}
        Sigue que $9|9(10^k+q)$, lo cual es verdadero. Por tanto, $9|10^n-1 \ \forall n\in \N$.\\
        Retomando que si $a_1,a_2,\dots, a_n$ son dígitos, el número cifrado $a_1a_2 \cdots a_n$ tiene la forma
        \begin{align*}
          &10^{n-1}a_1+10^{n-2}a_2+\dots+10^{0}a_n \\ &= 10^{n-1}a_1+10^{n-2}a_2+\dots+10^{0}a_n\\
        &= \big(10^{n-1}-1 \big)a_1+a_1+\big(10^{n-2}-1 \big)a_2+a_2+ \dots + \big(10^0-1 \big)a_n+a_n \\
        &= \big(10^{n-1}-1 \big)a_1+\big(10^{n-2}-1 \big)a_2+ \dots + \big(10^0-1 \big)a_n+ a_1+a_2 +\dots + a_n \end{align*}
        
        \pagebreak

        Sea $A=\big(10^{n-1}-1 \big)a_1+\big(10^{n-2}-1 \big)a_2+ \dots + \big(10^0-1 \big)a_n$ y $B=a_1+a_2 +\dots + a_n$.
        
        \begin{enumerate}[label=\roman*)]
            \item Si $9|a_1a_2\dots a_n$. Notemos que $9|10^n-1 \ \forall n\in\N$, y por \textbf{(P9)} $9|(10^n-1)c \ \forall c\in\Z$ y por \textbf{(P8)} $9|A$. Finalmente, por \textbf{(P13)} $9|B$, es decir, $9|a_1+a_2 +\dots + a_n$.
            \item Si $9|a_1+a_2+\dots a_n$. Notemos que $9|10^n-1 \ \forall n\in\N$, y por \textbf{(P9)} $9|(10^n-1)c \ \forall c\in\Z$. Por \textbf{(P8)} $9|A$ y por hipótesis, $9|B$. Finalmente, por \textbf{(P8)} $9|A+B$, es decir, $9|a_1a_2 \cdots a_n$.
          \end{enumerate}
    \end{enumerate}
\end{enumerate}

    \begin{enumerate}[label=30.1]
      \item Sea $d=(a_0,a_1,\dots,a_n)$. Si $d'=(a_0,a_1,\dots,a_n)$, por definición $d'|d$ y $d|d'$. Luego, por \textbf{(P6)} $d=d'$.
    \end{enumerate}

\begin{enumerate}[start=31]
  \item Sea $d=\big((a,b),c \big)$. Por definición, $d|(a,b)$ y $d|c$. También, $(a,b)|a$ y $(a,b)|b$. Luego, por \textbf{(P8)} $d|a$ y $d|b$. Luego, sea $d'\in\Z$ tal que $d'|a,b,c$, entonces $d'|(a,b)$. De este modo, $d'|d$. Por tanto, $((a,b),c)=(a,b,c)$.
  \item Sea $d=(a_1,a_2,\dots,a_n)$ y $d'=\big(|a_1|,|a_2|,\dots,|a_n| \big)$, entonces $d \big| a_1,a_2,\dots,a_n$ y $d' \Big||a_1|,|a_2|,\dots,|a_n|$. Por \textbf{(P12)} se verifica que $d \Big| |a_1|,|a_2|,\dots,|a_n|$ y $d' \big| a_1,a_2,\dots,a_n$, lo que implica que $d|d'$ y $d'|d$. Finalmente, por \textbf{(P6)} $d=d'$, es decir, $(a_1,a_2,\dots,a_n)=\big(|a_1|,|a_2|,\dots,|a_n| \big)$.
\end{enumerate}

\begin{enumerate}[label=33.1]
  \item Sea $m=[a,b]$ y $m'=[a,b]$, entonces $a|m$ y $b|m$. También, $a|m'$ y $b|m'$. Por definición, $m|m'$ y $m'|m$. Finalmente, por \textbf{(P6)} $m=m'$.
\end{enumerate}

\begin{enumerate}[label=33.2]
  \item Sea $A=\{x\in \N: \ a|x$ y $b|x\}$. Claramente, $a|ab$ y $b|ab$, entonces $\exists q_1, q_2\in \Z$ tales que $ab=aq_1$ y $ab=aq_2$. Luego,
    \begin{enumerate}[label=\roman*)]
      \item Si $ab \geq 0$, entonces $|ab|=ab$, por lo que $a \big| |ab|$ y $b \big| |ab|$.
      \item Si $ab <0$, entonces $|ab|=-ab$. Notemos que $-ab=-aq_1$ y $-ab=-bq_2$. Entonces, $|ab|=a(-q_1)$ y $|ab|=b(-q_2)$, lo que implica que $a \big |ab|$ y $a \big| |ab|$.
    \end{enumerate}
    De este modo, $|ab|\in A$. Así $A \neq \emptyset$ y por el principio del buen orden, $A$ contiene un elemento mínimo $m$. Supongamos que $\exists s$ tal que $a|s$ y $b|s$. Por el algoritmo de la división, $\exists q,r\in\Z$, únicos tales que
      \begin{align*}
        s &= mq+r \ \text{con $0 \leq r < |m|$}
      \end{align*}
      \\Sigue que, $r=s-mq$. Observemos que
      \begin{enumerate}[label=\roman*)]
        \item Como $a|m$ y $a|s$, $\exists k_1,k_2\in \Z$ tales que $m=ak_1$ y $s=ak_2$, luego $mq=ak_1q$. Notemos que $s-mq=ak_2-ak_1q$, es decir, $r=a(k_2-k_1q)$, lo cual implica que $a|r$.
        \item Como $b|m$ y $b|s$, $\exists k_3,k_4 \in \Z$ tales que $m=bk_3$ y $s=bk_4$, luego $mq=bk_4q$. Notemos que $s-m=bk_3-bk_4q$, es decir, $r=b(k_3-k_4q)$, lo cual implica que $b|r$.
      \end{enumerate}
      Tenemos que $m\in \N$, entonces $|m|=m$. Así $0 \leq r<m$. Observemos que si $0<r<m$, entonces $r\in A$ y se contradice que $m$ es elemento mínimo de $A$, por lo que $r=0$. Finalmente, $s=mq$. Por tanto $m|s$.
\end{enumerate}

\begin{enumerate}[label=33.3]
  \item Sea $m=[a,b]$ y $m'=\big[|a|,|b| \big]$, entonces $a|m$ y $b|m$. También $|a| \big| m'$ y $|b| \big| m'$. Por \textbf{(P12)} se verifica que $|a| \big| m$ y $|b| \big| m$. Por definición, $m|m'$ y $m'|m$. Finalmente, por \textbf{(P6)}, $m=m'$, es decir, $[a,b]=\big[|a|,|b| \big]$.
\end{enumerate}

\begin{enumerate}[label=33.4]
\item Sea $m=[a,b]$ y $d$ tal que $md=ab$. Tenemos que $a|m$, entonces $\exists s,t \in \Z$ tales que $m=as$ y $m=bt$. Luego, $md=asd$ y $md=btd$, entonces $ab=asd$ y $ab=btd$. Sigue que $a=td$ y $b=sd$, lo que implica que $d|a$ y $d|b$. Sea $d'\in\Z$ tal que $d'|a$ y $d'|b$, entonces $\exists a',b'\in \Z$ tales que $a=d'a'$ y $b=d'b'$. Definamos $m'=a'b'd'$, así tenemos $m'=ab'$ y $m'=a'b$, lo que implica que $a|m'$ y $b|m'$. Por definición, $m|m'$, entonces $\exists q\in \Z$ tal que $m'=mq$. Luego, $m'd'=mqd'$, es decir, $m'd'=a'b'd'd'$, osea $m'd'=ab$ y $m'd'=md$. De este moodo, $mqd'=md$, de donde se sigue que $qd'=d$, lo que implica que $d'|d$. Entonces, $d=(a,b)$. Finalmente, como $md=ab$, $[a,b](a,b)=ab$. Por tanto, $[a,b]=\frac{a \cdot b}{(a,b)}$.
\end{enumerate}

\begin{enumerate}[label=33.5]
  \item Notemos que
  \begin{align*}
    [ak,bk] &= \frac{ak \cdot bk}{(ak,bk)} && \text{Por el ejercicio 33.4}\\
    &= \frac{ak \cdot bk}{k(a,b)} && \text{Por el ejercicio 18}\\
    &= \frac{k^2 \cdot ab}{k(a,b)} \\
    &= k \frac{ab}{(a,b)}
  \end{align*}
Luego,
\begin{align*}
  [a,b] &= \frac{a \cdot b}{(a,b)} && \text{Por el ejercicio 33.4} \\
  k[a,b] &= k \frac{a \cdot b}{(a,b)} \\
\end{align*}
Por tanto, $[ak,bk]=k[a,b]$.
\end{enumerate}

\begin{enumerate}[start=37]
  \item Notemos que $p|a_1(a_2 \cdot \ldots \cdot a_n)$.
    \begin{gather*}
      \text{Si $p|a_1$ se cumple nuestra tesis. Si $p \nmid   a_1$, entonces $p|a_2 \cdot \ldots \cdot a_n$} \\
      \text{Si $p|a_2$ se cumple nuestra tesis. Si $p \nmid   a_2$, entonces $p|a_3 \cdot \ldots \cdot a_n$} \\
      \vdots \\
      \text{Si $p|a_{n-1}$ se cumple nuestra tesis. Si $p   \nmid a_{n-1}$, entonces $p|a_n$} \\
    \end{gather*}
    Por tanto, $p|a_i$ para algún $i=1,2,\dots,n$.
  \item Si $a<-1$, entonces $-a>1$. Por el teorema fundamental de la aritmética, $-a=p_1\cdot p_2 \cdot \ldots \cdot p_n$. Por tanto, $a=-p_1\cdot p_2 \cdots p_n$.
\end{enumerate}

\begin{enumerate}[label=40]
  \item Supongamos que $n$ no es primo, entonces $\exists m,k \in \Z^+ -\{0 \}$ con $m,k>1$ tales que $n=mk$. De este modo
    \begin{align*}
      2^n-1 &= 2^{mk}-1\\
      &= (2^k)^m-1^m\\
      &= (2^k-1) \Big( \big(2^k \big)^{m-1}+\big(2^k\big)^{m-2}+ \dots +\big(2^k\big)^0 \Big)\\
    \end{align*}
  Notemos que $m,k>1$, entonces $2^k-1 > 1$ y $\Big( \big(2^k \big)^{m-1}+\big(2^k\big)^{m-2}+ \dots +\big(2^k\big)^0 \Big) > 1$.\\ Sea $q_1=2^k-1$ y $q_2=\Big( \big(2^k \big)^{m-1}+\big(2^k\big)^{m-2}+ \dots +\big(2^k\big)^0 \Big)$. Entonces $2^n-1=q_1q_2$ con $q_1,q_2\in \Z^+ -\{0 \}$. Por tanto, $2^n-1$ no es primo.
\end{enumerate}

\begin{enumerate}[start=43]
  \item Tenemos que $p$ es primo, entonces los únicos divisores de $p^{n-1}$ son $p^{n-1}, p^{n-2}, \dots, 1$. Luego, $p^{n-1}=(p-1)\big( p^{n-1}+p^{n-2}+\dots +1 \big)$. De este modo, $\frac{p^{n-1}}{p-1}=\big( p^{n-1}+p^{n-2}+\dots +1 \big)$. Por tanto, la suma de divisores positivos de $p^{n-1}$ es igual a $\frac{p^{n-1}}{p-1}$.
  \item Supongamos que el conjunto de los números primos es finito. Definamos a $P$ como el conjunto de los números primos. Tenemos que $P=\{2,3, \ldots, p_n \}$ para algún $n\in\N$. Luego, sea $a=2 \cdot 3 \cdot \ldots \cdot p_n$, vemos que $2|a, 3|a, \ldots , p_n|a$. Tomemos $a+1$, para el cual tenemos que $2 \nmid a, 3 \nmid a, \ldots , p_n \nmid a$. Entonces, $a+1\in P$, lo cual es una contradicción. Por tanto, el conjunto de los números primos no es finito.
\end{enumerate}


%CAPÍTULO 1

\section*{Capítulo 1. Los números complejos}

%CONJUGADOS

\subsection*{Proposición 1.4.2} Si $z_1$ y $z_2$ son complejos, entonces: \begin{enumerate}[label=\roman*)]
  \item $\overline{(\overline{z_1})}=z_1$.
  \item $\overline{z_1+z_2} = \overline{z_1}+\overline{z_2}$.
  \item $\overline{z_1\cdot z_2} = \overline{z_1} \cdot \overline{z_2}$.
  \item $\overline{z_1-z_2} = \overline{z_1} - \overline{z_2}$.
\end{enumerate}

\textbf{Demostración}

Sea $z_1=a+bi$ t $z_2=c+di$. \begin{enumerate}[label=\roman*)] 
  \item \begin{align*}
    \overline{(\overline{z_1})} &= \overline{(\overline{a+bi})}\\
    &= \overline{(a-bi)} \\
    &= a-(-bi) \\
    &= a+bi\\
    &= z_1
  \end{align*}
  
  \item \begin{align*}
    \overline{z_1+z_2}  &= \overline{(a+bi)+(c+di)}\\
    &= \overline{(a+c)+(b+d)i}\\
    &= (a+c)-(b+d)i\\
    &= a-bi+c-di \\
    &= \overline{z_1}+\overline{z_2}
  \end{align*}

  \item \begin{align*}
    \overline{z_1\cdot z_2} &= \overline{(a+bi)\cdot (c+di)} \\
    &= \overline{(ac-bd)+(ad+bc)i}\\
    &= (ac-bd)-(ad+bc)i\\
    &= (ac-bd)+(-ad-bc)i\\
    &= (a-bi)\cdot (c-di)\\
    &= \overline{z_1} \cdot \overline{z_2}
  \end{align*}

  \item \begin{align*}
    \overline{z_1-z_2} &= \overline{(a+bi)-(c+di)} \\
    &= \overline{a+bi-c-di}\\
    &= \overline{(a-c)+bi-di}\\
    &= \overline{(a-c)+(b-d)i}\\
    &= (a-c)-(b-d)i\\
    &= a-c-bi+di\\
    &= (a-bi)-(c-di)\\
    &= \overline{z_1} - \overline{z_2}
  \end{align*}

\end{enumerate}

%MÓDULO

\subsection*{Observación.}

\begin{enumerate}[label=\roman*)]
  \item $z \cdot \overline{z} = a^2+b^2$.
  \item $|z| = \sqrt{z\cdot \overline{z}}$.
  \item $|z|^2 = z \cdot \overline{z}$.
  \item $z+\overline{z} = 2Re(z)$.
  \item $Re(z)\leq |z|$
  \item $\overline{z_1\cdot \overline{z_2}} = \overline{z_1} \cdot z_2$.
\end{enumerate}

\textbf{Demostración.}

\begin{enumerate}[label=\roman*)]
  \item \begin{align*}
    z \cdot \overline{z} &= (a+bi)(a-bi)\\
    &= a^2+b^2-abi+abi\\
    &= a^2+b^2
  \end{align*}
  
  \item \begin{align*}
    |z| &=\sqrt{a^2+b^2} \\
    &= \sqrt{z\cdot \overline{z}}\\
  \end{align*}

  \item \begin{align*}
    |z|^2 &= \left(\sqrt{z \cdot \overline{z}}\right)^2 \\
    &= z \cdot \overline{z}
  \end{align*}

  \item \begin{align*}
    z + \overline{z} &= (a+bi) + (a-bi) \\
    &= 2a
  \end{align*}

  \item $\forall a\in \R, a \leq |a|=\sqrt{a^2} \leq \sqrt{a^2+b^2}$, entonces

  \item \begin{align*} 
    \overline{z_1}\cdot \overline{z_2} &= \overline{z_1} \cdot \overline{\overline{z_2}} \\
    &= \overline{z_1} \cdot z_2
  \end{align*}

\end{enumerate}

\subsection*{Proposición 1.4.3} Si $z_1$ y $z_2$ son complejos, entonces: \begin{enumerate}[label=\roman*)]
  \item $|z_1|=0$, si y solo si, $z_1=0$.
  \item $|\overline{z_1}| =|z_1|$.
  \item $|z_1\cdot z_2|=|z_1|\cdot|z_2|$.
  \item $|z_1+z_2|\leq |z_1|+|z_2|$
  \item Si $z_2\neq 0$, $|\frac{z_1}{z_2}|= \frac{|z_1|}{|z_2|}$.
  \item $|z_1|-|z_2|\leq |z_1-z_2|$.
\end{enumerate}

\textbf{demostración}

\begin{enumerate}[label=\roman*)]
  \item \begin{enumerate}[label=\alph*)]
    \item \begin{align*}
      |z_1| &= 0\\
      |a+bi| &= 0\\
      \sqrt{a^2+b^2} &= 0 \\
      a^2+b^2 &= 0 \\
      a^2 &= -b^2 
    \end{align*}\\
    Como $a,b\in \R$ y $x^2\geq 0 \ \forall x\in \R$, sigue que $a=0$ y $b=0$, por tanto $z_1=0$.
    \item Si $z_1=0$, es inmediato que $|z_1|=0$.
  \end{enumerate}
  
  \item \begin{align*}
    |\overline{z_1}| &= |a-bi|\\
    &= |a+(-b)i| \\
    &= \sqrt{a^2+(-b)^2}\\
    &= \sqrt{a^2+b^2}\\
    &= |z_1|
  \end{align*}

  \item \begin{align*}
    |z_1 \cdot z_2|^2 &= (z_1 \cdot z_2) (\overline{z_1 \cdot z_2})\\
    &= (z_1 \cdot z_2) (\overline{z_1} \cdot \overline{z_2} )\\
    &= (z_1 \overline{z_1}) \cdot (z_2 \overline{z_2})\\
    &= |z_1|^2 \cdot |z_2|^2
  \end{align*}\\
  Notemos que $|z|\geq 0 \ \forall z\in \C$. Por tanto, $|z_1 \cdot z_2|=|z_1|\cdot |z_2|$.

  \item \begin{align*}
    |z_1+z_2|^2 &= (z_1+z_2) (\overline{z_1+z_2})\\
    &=  (z_1+z_2) (\overline{z_1} + \overline{z_2})\\
    &= z_1 \cdot \overline{z_1} + z_1\cdot \overline{z_2} + z_2 \overline{z_1} + z_2 \cdot \overline{z_2} \\
    &= |z_1|^2+ z_1\cdot \overline{z_2} + z_2 \cdot \overline{z_1} + |z_2|^2 \\
    &= |z_1|^2+ z_1\cdot \overline{z_2} + \overline{\overline{z_2}} \cdot \overline{z_1} + |z_2|^2 \\
    &= |z_1|^2+ z_1\cdot \overline{z_2} + \overline{\overline{z_2} \cdot z_1} + |z_2|^2 \\
    &= |z_1|^2+ 2 Re(z_1\cdot \overline{z_2}) + |z_2|^2
  \end{align*} \\
  Notemos que  

  \item 
\end{enumerate}

\end{document}
